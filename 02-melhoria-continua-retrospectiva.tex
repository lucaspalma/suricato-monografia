\section{Melhoria Contínua e Retrospectiva}

Com o passar do tempo, é comum que pessoas e times adotem certos padrões de como trabalhar juntos com menos atrito e de forma mais produtiva. No âmbito técnico, essa tendência se demonstra nas práticas de desenvolvimento utilizada. No pessoal, no formato da comunicação, sua frequência e do relacionamento entre as pessoas.

Como diz Aristóteles: 

\begin{quote}
\textit{``Só fazemos melhor aquilo que repetidamente insistimos em melhorar. A busca da excelência não deve ser um objetivo, e sim um hábito.''}
\end{quote}

Ou seja, sempre há espaço para melhorias. Assim, como não existe uma forma perfeita de se trabalhar, as pessoas e os projetos devem estar em uma constante evolução.

No estudo de processos de desenvolvimento de projetos, outra literatura que trata melhoria contínua com grande seriedade é o \textit{Lean}: a filosofia por trás do modelo Toyota de produção~\cite{lean}. O conceito recebe o nome de \textit{Kaizen} e, em uma empresa que seguem as ideias do \textit{Lean}, todas as pessoas, independentemente de salário ou hierarquia, são responsáveis por pensar e agir para melhoria contínua do processo e do produto oferecido.

Esses são apenas alguns demonstrativos da importância do princípio de melhoria contínua: é interessante que os times usem parte do seu tempo para pensar no que está dando certo ou errado, quais áreas precisam de aperfeiçoamento e que ações podem ser tomadas para levar o time a uma evolução. Equipes que conseguem fazer isso, a longo prazo, aumentam a qualidade do código, entregam mais valor para o cliente, criam um ambiente de trabalho melhor, entre outros benefícios.

Aplicar os conceitos de melhoria contínua no dia a dia, contudo, não é uma tarefa trivial. Há diversas práticas e técnicas que podem servir para esse propósito e, no contexto dos métodos ágeis, a ferramenta mais conhecida com esse objetivo é a retrospectiva.

\subsection{Estrutura e papéis em retrospectivas}

Dependendo do contexto, uma retrospectiva pode se tornar uma reunião complexa, dividida em uma série de etapas e é comum que os participantes percam a visão do todo e se percam em algum problema. Assim, é interessante e muito comum que uma pessoa assuma um papel mais neutro, focando no andamento e nas atividades da reunião. Essa pessoa é chamada de \textbf{facilitador}.

Sua função é manter o time focado, sem desfocar durante discussões ou se alongar demais em temas pouco relevantes. Ele pode fazer isso, por exemplo, relembrando o assunto principal, delimitando tempos máximos para cada assunto e estruturando as atividades que comporão a reunião.

Durante a retrospectiva, o time deve ser capaz de identificar obstáculos e situações adversas. Esse é o momento para identificar problemas, definir ações e responsáveis por fazer cada ação ser cumprida e, 	assim, por fazer mudanças acontecerem na equipe, dividindo assim a responsabilidade.

Para organizar tudo isso, há diversos trabalhos a respeito de formatos para essa reunião. A sugestão mais conhecida entre agilistas é o formato apresentado por Derby e Larsen~\cite{retrospectives} no livro Agile Retrospectives: 

\begin{enumerate}
	\item Preparar o terreno: 5\% do tempo.
	\item Coletar os dados: 30\% a 50\% do tempo.
	\item Gerar ideias: 20\% a 30\% do tempo.
	\item Decidir o que fazer: 15\% a 20\% do tempo.
	\item Encerrar a retrospectiva: 10\% do tempo.
\end{enumerate}

Para realizar estas 5 etapas, o facilitador escolhe um modelo de retrospectiva para ser usado, conhecido como atividade. Existem muitos livros que catalogam diversas atividades conhecidas e utilizadas por vários times ágeis para realizar a reunião, como \textit{Fun Retrospectives}~\cite{funRetrospectives}, \textit{Retrospective Handbook}~\cite{handRetrospectives} e \textit{Agile Retrospectives}~\cite{retrospectives}.

\subsubsection*{Preparar o terreno}

O objetivo desta etapa é ajudar o time a se concentrar no que será feito no decorrer da reunião e passar para os participantes qual será o objetivo da retrospectiva.

Além disso, esta etapa também é responsável por estabelecer o ambiente no qual as pessoas discutirão. Por isso, é necessário estabelecer uma atmosfera que deixe as pessoas confortáveis para debaterem e expressarem suas opiniões.

\subsubsection*{Coletar os dados}

Mesmo em iterações curtas de uma ou duas semanas, é muito difícil que todas as pessoas tenham visto e participado de tudo que aconteceu. Por isso, é importante que todos criem uma visão uniforme sobre o que aconteceu no projeto, caso contrário cada um baseará seus pensamentos apenas nas suas próprias opiniões e pontos de vista sobre a última iteração, tornando difícil estabelecer ações que todos se comprometerão. Desta forma, esta etapa busca estabelecer uma mesma imagem do que andou acontecendo para todos os membros. Para formar esta imagem será necessário coletar dados sobre a iteração.

Existem dois tipos de dados a serem coletados. Os fatos, como eventos, métricas, funcionalidades, histórias terminadas e assim por diante. A análise destes pode ajudar as pessoas a encontrar padrões e fazer ligações entre os últimos acontecimentos. E os dados subjetivos, que mostram a importância, para cada um dos integrantes, a respeito dos fatos e da equipe. 

Por fim, baseados em ambos os dados apresentados, são coletados os pontos fortes e fracos da equipe na iteração.

\subsubsection*{Gerar ideias}

Depois que um ambiente propício para discussões é estabelecido e todos estão olhando para os mesmo dados, é o momento de dar um passo pra trás para visualizar a imagem que foi gerada e pensar no que fazer. 

Esta é a etapa em que as pessoas investigam quais foram os problemas enfrentados e os riscos que elas estão correndo. A partir destas discussões propõem diversas ideias sobre o que pode ser feito para que o time melhore.

\subsubsection*{Decidir o que fazer}

Após obter um conjunto de ideias, é o momento de listar quais são as melhores ações e escolher aquelas que serão executadas.

É comum nesta etapa as pessoas apontarem muitas ações e quererem adotar todas. No entanto, acaba sendo bem difícil que todas sejam incorporadas e, algumas vezes, nenhuma delas é utilizada. Então o ideal é que ao final desta etapa sejam eleitas poucas ações que as pessoas vão conseguir adotar e que trarão algum benefício.

Além disso, é interessante delegar a alguém a responsabilidade por realizar a ação planejada -- quando ninguém fica como responsável é bem comum que a mudança não seja feita.

\subsubsection*{Encerrar a retrospectiva}

No decorrer da retrospectiva o time esteve, a todo momento, discutindo e tendo ideias. Com isto vem também um aprendizado para as pessoas sobre a equipe e o projeto. No entanto, nem todas as ideias que foram apresentadas serão utilizadas na próxima iteração.

Além disso, o facilitador pode  buscar os pontos positivos e negativos da reunião promovendo, dessa forma, uma breve retrospectiva da retrospectiva. Assim, haverá \textit{feedback} para que o facilitador melhore sua atuação e a escolha de atividades para a próxima vez.

\subsection{Ferramentas usadas em retrospectivas}

Uma forma utilizada para realizar a retrospectiva é usando uma lousa para apresentar a atividade da reunião e post-its para coletar os dados. Este processo funciona para times cujos membros trabalham no mesmo local. Para as equipes em que as pessoas estão distribuídas em diferentes regiões, existem ferramentas que simulam a estrutura da reunião local.

A Retrium~\footnote{https://retrium.com/} oferece um serviço web para times realizarem a reunião seguindo uma estrutura próxima da apresentada por Derby e Larsen. Ele começa com a escolha de uma atividade, seguido da coleta dos dados, agrupamento de informações comuns, priorização dos comentários mais relevantes e termina com a geração das ideias.

O principal problema da aplicação é o baixo número de atividades oferecidas. Atualmente ela suporta apenas 3 formatos conhecidos de retrospectiva:

\begin{itemize}
	\item Mad, Sad and Glad
	\item Start, Stop and Continue
	\item Liked, Learned, Lacked and Longed for
\end{itemize}

Todos estes formatos seguem o princípio de discutir apenas os problemas da equipe, ignorando retrospectivas que discutem, por exemplo, o futuro ou a cultura do time. 

Outro problema é a falta de flexibilidade do sistema que exige que o usuário siga o mesmo fluxo para a reunião. Usuários que já estão acostumados com o processo da retrospectiva costumam adaptar a reunião para o seu cenário, o que não é possível fazer através da Retrium.

A GroupMap~\footnote{http://www.groupmap.com/map-templates/agile-retrospective/} oferece um serviço que trabalha apenas com as atividades mais simples, que verificam apenas os pontos positivos e negativos. Os problemas são bem parecidos com os encontrados na Retrium, como a falta de flexibilidade no fluxo da reunião. Os times não habituados com a retrospectiva também podem achar o sistema complexo, uma vez que não há marcações ou dicas sobre o que são cada um dos recursos.

Essas aplicações focam apenas na realização da retrospectiva e atinge apenas os times que já sabem realizá-la. No entanto, existem problemas que não são cobertos por essas aplicações.