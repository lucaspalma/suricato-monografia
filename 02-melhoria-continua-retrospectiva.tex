\section{Melhoria Contínua e Retrospectiva}

Com o passar do tempo, é comum que pessoas e times adotem certos padrões para conseguirem trabalhar juntos com menos atrito e de forma mais produtiva. No âmbito técnico, essa tendência se demonstra nas práticas de desenvolvimento utilizada. No pessoal, no formato da comunicação, sua frequência e relacionamento entre as pessoas envolvidas.

Como disse Aristóteles: 

\begin{quote}
\textit{``Só fazemos melhor aquilo que repetidamente insistimos em melhorar. A busca da excelência não deve ser um objetivo, e sim um hábito.''}
\end{quote}

Ou seja, sempre há espaço para melhorias. Assim, como não existe uma forma perfeita de se trabalhar, pessoas, processos e projetos devem estar em constante evolução.

No estudo de processos de desenvolvimento de projetos, outra literatura que trata melhoria contínua com grande seriedade é a \textit{Lean}: a filosofia por trás do modelo Toyota de produção~\cite{lean}. O conceito recebe o nome de \textit{Kaizen} e, em uma empresa que seguem as ideias do \textit{Lean}, todas as pessoas, independentemente de salário ou hierarquia, são responsáveis por pensar e agir para melhoria contínua do processo e do produto oferecido.

Esses são apenas alguns demonstrativos da importância do princípio de melhoria contínua: é interessante que os times usem parte do seu tempo para pensar no que está dando certo ou errado, quais áreas precisam de aperfeiçoamento e que ações podem ser tomadas para levar o time a uma evolução. Equipes que conseguem fazer isso, a longo prazo, aumentam a qualidade do código, entregam mais valor para o cliente e criam um ambiente de trabalho melhor, entre outros benefícios.

Aplicar os conceitos de melhoria contínua no dia a dia, contudo, não é uma tarefa trivial. Há diversas práticas e técnicas que podem servir para esse propósito e, no contexto dos métodos ágeis, a ferramenta mais conhecida com esse objetivo é a retrospectiva.

\subsection{Estrutura e papéis em retrospectivas}

Dependendo do contexto, uma retrospectiva pode se tornar uma reunião complexa, dividida em uma série de etapas e é comum que os participantes percam a visão do todo e se percam em algum problema. Assim, é interessante e muito comum que uma pessoa assuma um papel mais neutro, focando no andamento e nas atividades da reunião. Essa pessoa é chamada de \textbf{facilitador}.

Sua função é manter o time focado, sem dispersar durante discussões ou se alongar demais em temas pouco relevantes. Ele pode fazer isso, por exemplo, relembrando o assunto principal, delimitando tempos máximos para cada assunto, ou estruturando as atividades que comporão a reunião.

Durante a retrospectiva, o time deve ser capaz de identificar obstáculos e situações adversas. Esse é o momento para identificar problemas, definir ações e responsáveis por fazer cada ação ser cumprida e, assim, tornar todo o time responsável por fazer melhorias acontecerem.

Para organizar tudo isso, há diversas publicações sugerindo formatos para essa reunião. O formato apresentado por Derby e Larsen~\cite{retrospectives} no livro Agile Retrospectives é o mais conhecido entre agilistas:

\begin{enumerate}
	\item Preparar o terreno: 5\% do tempo.
	\item Coletar os dados: 30\% a 50\% do tempo.
	\item Gerar ideias: 20\% a 30\% do tempo.
	\item Decidir o que fazer: 15\% a 20\% do tempo.
	\item Encerrar a retrospectiva: 10\% do tempo.
\end{enumerate}

Para realizar a retrospectiva com essas 5 etapas, o facilitador pode inventar atividades que façam esse papel, ou se valer de diversas técnicas já catalogadas em páginas da internet ou livros, tais quais \textit{Fun Retrospectives}~\cite{funRetrospectives}, \textit{Retrospective Handbook}~\cite{handRetrospectives} e \textit{Agile Retrospectives}~\cite{retrospectives}.

\subsubsection*{Preparar o terreno}

O objetivo desta etapa é ajudar o time a se concentrar no que será feito no decorrer da reunião e passar para os participantes qual será o objetivo da retrospectiva.

Além disso, é nesta etapa que se estabelece o ambiente no qual as pessoas discutirão. Para garantir que a reunião focará na melhoria do time, é necessário garantir que cada pessoa se sinta confortável para expressar suas opiniões e contribuir com as discussões.

\subsubsection*{Coletar os dados}

Mesmo com iterações curtas, tipicamente de uma ou duas semanas, e times pequenos, de até dez pessoas, é improvável que todos tenham conhecimento e participação em tudo que aconteceu com o time.

Antes de decidir o que fazer, é importante que todos criem uma visão uniforme sobre o que aconteceu no projeto recentemente, levando em conta as opiniões e pontos de vista de cada pessoa do time. A etapa de coleta de dados busca expor as opiniões e visões de cada membro do time para estabelecer uma mesma imagem dos acontecimentos que estão em discussão.

Existem dois tipos de dados a serem coletados: fatos e aspectos subjetivos. Os fatos, são eventos, métricas, funcionalidades, histórias terminadas, e demais itens tangíveis. A análise destes pode ajudar as pessoas a encontrar padrões e fazer ligações entre os últimos acontecimentos. Os dados subjetivos, são as impressões, opiniões, julgamentos de importância e valor que cada integrantes atribui aos fatos.

O levantamento dessas informações, baseados em ambos os tipos de dados apresentados, leva a uma visão maior sobre o que está correndo bem e o que precisa de ações para melhoria.

\subsubsection*{Gerar ideias}

Depois que um ambiente propício para discussões é estabelecido e todos estão olhando para os mesmo dados, é o momento de dar um passo pra trás para visualizar a imagem que foi gerada e pensar no que fazer. 

Esta é a etapa em que as pessoas investigam quais foram os problemas enfrentados, quais vitórias alcançaram e os riscos a que projeto e time estão suscetíveis. A partir destas discussões, todos os membros do time propõem diversas ações sobre o que pode ser feito em busca de melhorias.

A fase de gerar ideias pode ser executada de forma orgânica, quando todos os membros se sentirem confortáveis em lançar ideias livremente, ou de forma mais estruturada e impessoal, quando a equipe ainda não está madura. Técnicas como \textit{brainstorming} e \textit{brainwriting}~\cite{brainwriting}, termos utilizados em inglês mesmo na literatura brasileira sobre o tema, são bastante comuns nessa fase.

\subsubsection*{Decidir o que fazer}

Após obter um conjunto de ideias, chega o momento de selecionar as mais promissoras, transformá-las em ações e escolher aquelas que o time se compromete a executar.

Como as ideias vieram de todos, é comum que as pessoas apontem muitas ações e queiram adotar todas. No entanto, a literatura e especialistas em agilidade indicam índices melhores de sucesso na execução das ações de retrospectivas quando o time se compromete com um número pequeno delas, preferindo as de melhor relação custo-benefício.

Além disso, é interessante delegar a alguém a responsabilidade por realizar a ação planejada -- quando ninguém fica como responsável é bem comum que a mudança não seja feita.

Ao final da fase de decidir o que fazer, portanto, temos um pequeno conjunto de ações que buscam promover melhorias, já atribuidas a indivíduos ou sub-grupos que ficarão responsáveis pelo seu cumprimento.

\subsubsection*{Encerrar a retrospectiva}

No decorrer da retrospectiva o time esteve, a todo momento, discutindo e propondo ideias. Embora nem todas as ideias que foram apresentadas sejam selecionadas, existe valor no simples fato de todos terem tido sua chance de expor o que observaram e como se sentiram. Essa reunião cria, naturalmente, um maior senso de time. Para manter tal sensação, é comum que o facilitador separe um tempo para lembrar ao time desse fato.

Além disso, o facilitador pode buscar os pontos positivos e negativos sobre a própria reunião promovendo, dessa forma, uma breve retrospectiva da retrospectiva. Assim, haverá \textit{feedback} para que a próxima reunião seja ainda melhor conduzida e a escolha de atividades também se adeque mais às preferências do time para a próxima vez.

\subsection{Ferramentas usadas em retrospectivas}

Uma forma utilizada para realizar a retrospectiva é usando uma lousa para apresentar a atividade da reunião e post-its para coletar os dados. Este processo funciona para times cujos membros trabalham no mesmo local. Para as equipes em que as pessoas estão distribuídas em diferentes regiões, existem ferramentas que simulam a estrutura da reunião local.

A Retrium~\footnote{https://retrium.com/} oferece um serviço web para times realizarem a reunião seguindo uma estrutura próxima da apresentada por Derby e Larsen~\cite{retrospectives}. Ele começa com a escolha de uma atividade, seguido da coleta dos dados, agrupamento de informações comuns, priorização dos comentários mais relevantes e termina com a geração das ideias.

O principal problema da aplicação é o baixo número de atividades oferecidas. Atualmente ela suporta apenas 3 formatos conhecidos de retrospectiva:

\begin{itemize}
	\item Mad, Sad and Glad
	\item Start, Stop and Continue
	\item Liked, Learned, Lacked and Longed for
\end{itemize}

Todos estes formatos seguem o princípio de discutir apenas os problemas da equipe, ignorando retrospectivas que discutem, por exemplo, o futuro ou a cultura do time. 

Outro problema é a falta de flexibilidade do sistema que exige que o usuário siga o mesmo fluxo para a reunião. Usuários que já estão acostumados com o processo da retrospectiva costumam adaptar a reunião para o seu cenário, o que não é possível fazer através da Retrium.

A GroupMap~\footnote{http://www.groupmap.com/map-templates/agile-retrospective/} oferece um serviço que trabalha apenas com as atividades mais simples, que verificam apenas os pontos positivos e negativos. Os problemas são bem parecidos com os encontrados na Retrium, como a falta de flexibilidade no fluxo da reunião. Os times não habituados com a retrospectiva também podem achar o sistema complexo, uma vez que não há marcações ou dicas sobre o que são cada um dos recursos.

Essas aplicações focam apenas na realização da retrospectiva e atinge apenas os times que já sabem realizá-la. No entanto, existem problemas que não são cobertos por essas aplicações.