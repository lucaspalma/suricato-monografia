\section{Melhoria Contínua e Retrospectiva}

\begin{quote}
\textit{``At regular intervals, the team reflects on how to become more effective, then tunes and adjusts its behavior accordingly.''}~\cite{manifesto}
\end{quote}

Esta frase retrata um dos mais importantes princípios definidos no manifesto e uma das bases da agilidade, a melhoria contínua. 

Com o passar do tempo, é comum que as pessoas e os times adotem padrões de como trabalhar, seja no ambito técnico de quais práticas de desenvolvimento utilizar ou no pessoal de como é a comunicação e o relacionamento entre as pessoas. Como diz Stephen Hawking:

\begin{quote}
\textit{``One of the basic rules of the universe is that nothing is perfect. Perfection simply doesn't exist. Without imperfection, neither you nor I would exist''}
\end{quote}

Ou seja, quando se trata de pessoas e projetos que estão em constante evolução, não existe uma forma perfeita de se trabalhar. Sempre há espaço para melhorias.

Por isso da importância do princípio de melhoria contínua, já que é interessante que os times usem parte do seu tempo para pensar no que está dando certo ou errado, quais áreas precisam de aperfeiçoamento e que ações podem ser tomadas para levar o time a uma evolução. Equipes que conseguem fazer isso, a longo prazo, aumentam a qualidade do código, entregam mais valor para o cliente, criam um ambiente para o trabalho mais adequado, entre outros.

A melhoria contínua é o conceito do que os time devem fazer, mas como devem fazer isso? Promover o princípio não é algo trivial e ao longo do tempo surgiram práticas a mais conhecida é a retrospectiva.

No entanto, como os times promovem a melhoria contínua? Como dito anteriormente, ela é um princípio, ou seja, apenas o conceito do que deve ser feito. E como se faz isso? Com o passar do tempo, surgiram algumas práticas que aplicam esse princípio em um time ágil. A mais conhecida é a retrospectiva.

\subsection{Como funciona uma retrospectiva}

A retrospectiva pode ser um reunião complicada, dividida em uma série de etapas, e dado o tempo que ela leva, é comum as pessoas se perderem nos problemas. Assim, é interessante ter uma pessoa focada no andamento da reunião e não nos problemas. 

Essa pessoa é chamada de facilitador. Sua função é manter o time focado, sem se perder durante as discussões e se alongar demais em alguns temas. Ele pode fazer isso, por exemplo, relembrando o assunto principal e tentando evitar discussões de pouco valor.

Durante a retrospectiva, o time deve ser capaz de identificar obstáculos e situações adversas. Esse é o momento para resolver os problemas, definir ações e encontrar responsáveis por fazer as mudanças acontecerem na equipe, dividindo assim a responsabilidade.

Para organizar tudo isso, há diversas literaturas a respeito de formatos para essa reunião, como citados por Caroli e Caetano no livro Fun Retrospectives~\cite{funRetrospectives}, ou por Kua em Retrospective Handbook~\cite{handRetrospectives}. A sugestão mais conhecida entre agilistas é, no entanto, o formato apresentado por Derby e Larsen no livro Agile Retrospectives~\cite{retrospectives}: 

\begin{enumerate}
	\item Preparar o terreno: 5\% do tempo.
	\item Coletar os dados: 30\% a 50\% do tempo.
	\item Gerar ideias: 20\% a 30\% do tempo.
	\item Decidir o que fazer: 15\% a 20\% do tempo.
	\item Encerrar a retrospectiva: 10\% do tempo.
\end{enumerate}

Para realizar estas 5 etapas, o facilitador escolhe um modelo de retrospectiva para ser usado, conhecido como atividade. Livros como Fun Retrospectives~\cite{funRetrospectives} e Agile Retrospectives~\cite{retrospectives} apresentam catálogos com diversas atividades conhecidas e utilizadas por vários times ágeis para realizar cada uma das 5 etapas.

\subsubsection*{Preparar o terreno}

O objetivo desta etapa é ajudar o time a se concentrar no que será feito no decorrer da reunião e passar para as pessoas qual será o objetivo da retrospectiva.

Além disso, esta etapa também é responsável por estabelecer o ambiente no qual as pessoas discutirão. Por isso, é necessário estabelecer uma atmosfera que deixe as pessoas confortáveis para debaterem e expressarem suas opiniões.

\subsubsection*{Coletar os dados}

Mesmo em iterações curtas de uma ou duas semanas, é muito difícil que todas as pessoas tenham visto e participado de tudo que aconteceu. Por isso, é importante que todos criem uma visão uniforme sobre o que aconteceu no projeto, caso contrário cada um baseará seus pensamentos apenas nas suas próprias opiniões e pontos de vista sobre a última iteração, tornando difícil estabelecer ações que todos se comprometerão. Desta forma, esta etapa busca estabelecer uma mesma imagem do que andou acontecendo para todos os membros. Para formar esta imagem será necessário coletar dados sobre a iteração.

Os primeiros dados coletados são os fatos, como eventos, métricas, funcionalidades, histórias terminadas e assim por diante. Mostrando visualmente estes dados, as pessoas podem encontrar padrões e fazer ligações entre os últimos acontecimentos. Depois, são coletados os sentimentos. Eles mostram o que é importante para cada um dos integrantes a respeito dos fatos e da equipe.

Por fim, baseados nos fatos e sentimentos apresentados, são coletados os pontos fortes e fracos da equipe na iteração.

\subsubsection*{Gerar ideias}

Depois que um ambiente propício para discussões é estabelecido e todos estão olhando para os mesmo dados, é o momento de dar um passo pra trás para visualizar a imagem que foi gerada e pensar no que fazer. Então é nesta etapa em que as pessoas investigam quais foram os problemas enfrentados e os riscos que elas estão correndo. A partir destas discussões propõem diversas ideias sobre o que pode ser feito para que o time melhore.

\subsubsection*{Decidir o que fazer}

Após obter um conjunto de ideias, é o momento de listar quais são as melhores ações e escolher aquelas que serão executadas.

É comum nesta etapa as pessoas apontarem muitas ações e quererem adotar todas. No entanto, acaba sendo bem difícil que todas sejam incorporadas e, algumas vezes, nenhuma delas é utilizada. Então o ideal é que ao final desta etapa sejam eleitas uma ou duas ações que as pessoas vão conseguir adotar e que trarão algum benefício.

Além disso, é interessante delegar a alguém a responsabilidade por realizar a ação planejada -- quando ninguém fica como responsável é bem comum que a mudança não seja feita.

\subsubsection*{Encerrar a retrospectiva}

No decorrer da retrospectiva o time esteve, a todo momento, discutindo e tendo ideias. Com isto vem também um aprendizado para as pessoas sobre a equipe e o projeto. No entanto, nem todas as ideias que foram apresentadas serão utilizadas na próxima iteração.

Além disso, o facilitador pode  buscar os pontos positivos e negativos da reunião promovendo, dessa forma, uma breve retrospectiva da retrospectiva. Assim, haverá feedback para que o facilitador melhore sua atuação e a escolha de atividades para a próxima vez.