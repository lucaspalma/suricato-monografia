\section{Pesquisa}

Tanto a melhoria contínua quanto a retrospectiva já são conceitos muito conhecidos e utilizados pela comunidade ágil. Como cada time possui suas próprias necessidades e dificuldades, é comum a equipe adaptar estes conceitos para a sua situação. Isto dificulta para encontrar informações sobre como a melhoria contínua é aplicada atualmente e saber quais são as dificuldades que os times ágeis enfrentam ao realizar estas reuniões, principalmente a retrospectiva. 

Alguns destes dados só foram obtidos a partir de uma revisão sistemática da literatura sobre retrospectivas e quais são os problemas em seu desenvolvimento. No entanto, a maior parte do material que existe hoje foca na aplicação da reunião e nos passos durante a retrospectiva.

Os artigos que mostram o estado atual dos times ágeis consistem em relatórios de questionários que indicam a quantidade de times que fazem retrospectivas, como as pesquisas da VersionOne~\cite{versionOne} e da Scrum Alliance~\cite{scrumAlliance}. Em geral, os relatórios mostram a porcentagem de times que estão em cada etapa da adoção dos métodos ágeis, desde a introdução dessas metodologias na equipe até sua total adequação. Eles também mostram dados como a distribuição demográfica dos times que adotaram agilidade e quais métricas as equipes utilizam.

Boa parte das pesquisas apontavam que o uso de retrospectivas vem constantemente aumentando nos últimos anos. No entanto, são poucos os trabalhos que apresentam as dificuldades enfrentadas pelos times e, em geral, o assunto é tratado apenas para equipes locais. Existem duas possibilidades para essa falta de informação sobre equipes remotas: times usarem outras formas de promover melhoria contínua ou não haver publicações sobre o tema na comunidade.

Devido a falta de material de apoio para levantar estas informações, foi feita uma pesquisa sobre as práticas utilizadas por times locais e distribuídos, e quais desafios eles enfrentavam.

Para abranger uma parte ainda maior da comunidade ágil e enviesar menos o questionário em relação à ferramenta usada para trazer o princípio da melhoria contínua à tona, o assunto da pesquisa seria não apenas retrospectivas, mas o tópico maior: como times ágeis promovem melhoria de forma contínua.

\subsection{Motivação e questões da pesquisa}

Considerando, então, que o tema da pesquisa é o princípio de melhoria contínua, as questões foram imparciais de modo a não direcionar as respostas em torno apenas das retrospectivas. A parte inicial foca em questões gerais, como o número de integrantes, a distribuição da equipe e quais os desafios enfrentados. Depois são investigadas as particularidades sobre qual prática cada time utiliza para solucionar esses problemas e promover a melhoria contínua. 

Como a retrospectiva ainda é utilizada por diversos métodos ágeis e muitos times, parte do questionário é voltado apenas para as equipes que realizam esta reunião. As questões desta etapa buscam encontrar como cada time estrutura a reunião, quais os principais problemas que as equipes enfrentam na retrospectiva e a relação deles com a estrutura de cada time. 

Além disso, os artigos encontrados previamente já indicam que, embora nem todos os times façam a retrospectiva, eles ainda adotam alguma outra forma de promover a melhoria contínua. A última parte da pesquisa foca justamente nesses times que utilizam práticas diferentes da retrospectiva. Para estes casos, o questionário procura quais são as outras opções que são adotadas, como é o processo para realizar cada uma e os motivos que levaram o time a usar uma prática diferente da retrospectiva.

\subsection{Resultados}

O tempo para a coleta de respostas foi de 1 mês e foram obtidas 44 respostas. A pesquisa era composta de respostas quantitativas, com o intuito de coletar os problemas enfrentados e a utilização de retrospectivas, e qualitativas, que buscavam entender como os times resolviam seus problemas.

\subsubsection*{Resultados das perguntas quantitativas}

A primeira parte da pesquisa indicou que diferentes estruturas de equipes apontam diferentes problemas. A separação foi feita em: 
  
\begin{itemize}
  \item Local: times que estão no mesmo ambiente
  \item Distribuído no mesmo \textit{timezone}: times que não estão no mesmo ambiente, mas no mesmo fuso horário.
  \item Distribuído em \textit{timezones} diversos: times que não estão no mesmo ambiente e estão distribuídos em diferentes fusos horários.
\end{itemize}

Em seguida, são identificados os problemas que cada uma das equipes enfrentam, como, por exemplo, comunicação, relacionamento e produtividade. Como cada uma das estruturas possui as suas singularidades, os desafios enfrentados potencialmente estão relacionados com a distribuição dos integrantes. A análise desses dois dados em conjunto segue na tabela abaixo.

\begin{table}[H]
  \begin{adjustwidth}{-3em}{}
    \begin{tabular}{  m{6em}  m{3em} | m{6em} | m{7em} | m{6em} | m{5em} | m{5em} | }
      \cline{3-7} & & \multicolumn{5}{ c| }{Quais problemas seu time enfrenta? } \\ 
      \cline{1-7} \multicolumn{1}{ |m{6em}| }{Estrutura majoritária do time} & Total & Comunicação & Relacionamento & Produtividade & Nenhum & Outros \\
      \cline{1-7} \multicolumn{1}{ |m{6em}| }{Local} & 27 & 15 (55\%) & 4 (14\%) & 18 (66\%) & 4 (14\%) & 2 (7\%) \\ 
      \cline{1-7} \multicolumn{1}{ |m{6em}| }{Distribuído no mesmo \textit{timezone}} & 11 & 5 (45\%) & 1 (9\%) & 4 (36\%) & 3 (27\%) & 1 (9\%) \\
      \cline{1-7} \multicolumn{1}{ |m{6em}| }{Distribuído em \textit{timezones} diversos} & 6 & 4 (66\%) & 2 (33\%) & 1 (16\%) & 0 (0\%) & 3 (50\%) \\
      \cline{1-7} \multicolumn{1}{ |m{6em}| }{Total} & 44 & 24 (54\%) & 7 (15\%) & 23 (52\%) & 7 (15\%) & 6 (13\%) \\
      \cline{1-7}
    \end{tabular}
  \end{adjustwidth}
\end{table}

O desafio mais enfrentado por times ágeis é a comunicação. A pesquisa mostra que aproximadamente 54\% das equipes passam por esse problema. Além disso, os times distribuídos em \textit{timezones} diversos tendem a sofrer mais com este problema, já que 66\% deles passam por esta dificuldade, enquanto que os percentuais dos locais e distribuídos no mesmo \textit{timezone} foram de 55\% e 45\%, respectivamente.
  
Outro desafio no qual os times distribuídos em diversos fusos horários aparentam ter mais dificuldades do que as outras equipes é a de relacionamento. Aproximadamente 33\% dessas equipes relataram que passam por esse problema, contra apenas 14\% dos locais e 9\% dos distribuídos no mesmo \textit{timezone}.
  
Apesar dos times distribuídos em diversos fusos terem mais problemas de comunicação e relacionamento, eles possuem o melhor rendimento entre as três estruturas. Apenas 16\% deles possuem problemas de produtividade. Já para os times distribuídos no mesmo \textit{timezone} o valor chegou a 36\%, ou seja, mais do que o dobro. Porém, os times que mais sofrem com este problema são os locais que apresentaram um percentual de 66\%.
  
Por fim, os times distribuídos em \textit{timezones} diversos apresentaram outros problemas que não foram apontados pelas outras estruturas. A pesquisa mostrou que eles tem dificuldade para sincronizar todos os seus membros. Como as pessoas trabalham em fusos horários diferentes fica difícil definir as datas para os diversos tipos de reuniões ágeis, como o \textit{daily}, o \textit{planning}, a \textit{review} e a retrospectiva.

Após as questões sobre os desafios que cada um dos tipos de times ágeis enfrentam, os próximos resultados mostram como esses problemas são resolvidos e de que forma a melhoria contínua é promovida. Diversas pesquisas e artigos da área indicam que a grande maioria das equipes adotam a retrospectiva como uma das práticas ágeis utilizadas. A pesquisa apenas confirmou essa hipótese e mostrou como é sua aceitação pelos diversos tipos de equipes.

\begin{table}[H]
  \centering
  \begin{tabular}{  m{6em}  m{3em} | m{6em} | m{7em} |}
    \cline{3-4} & & \multicolumn{2}{ c| }{Seu time realiza retrospectivas?} \\ 
    \cline{1-4} \multicolumn{1}{ |m{6em}| }{Estrutura majoritária do time} & Total & Sim & Não \\
    \cline{1-4} \multicolumn{1}{ |m{6em}| }{Local} & 27 & 21 (77\%) & 6 (22\%) \\
    \cline{1-4} \multicolumn{1}{ |m{6em}| }{Distribuído no mesmo \textit{timezone}} & 11 & 9 (81\%) & 2 (18\%) \\
    \cline{1-4} \multicolumn{1}{ |m{6em}| }{Distribuído em \textit{timezones} diversos} & 6 & 1 (16\%) & 5 (83\%) \\
    \cline{1-4} \multicolumn{1}{ |m{6em}| }{Total} & 44 & 31 (70\%) & 13 (29\%) \\
    \cline{1-4}
  \end{tabular}
\end{table}

Aproximadamente 70\% dos participantes responderam que realizam retrospectivas. No entanto, é interessante notar como foram as respostas para cada uma das estruturas das equipes. Os times locais e distribuídos no mesmo \textit{timezone} são os que mais utilizam desta reunião como prática de melhoria contínua. Em contrapartida, aqueles que são distribuídos em \textit{timezones} diversos não costumam realizar retrospectivas já que apenas 16\% responderam que sim.

Um questão que ainda não foi comprovada pelos artigos na comunidade é se a retrospectiva realmente auxilia os times a resolver seus problemas. A pesquisa mostrou os seguintes resultados. 

\begin{table}[H]
  \begin{adjustwidth}{-3em}{}
    \begin{tabular}{ m{6em} m{3em} | m{6em} | m{7em} | m{6em} | m{5em} | m{5em} | }
      \cline{3-7} & & \multicolumn{5}{ c| }{Quais problemas seu time enfrenta?} \\ 
      \cline{1-7} \multicolumn{1}{ |m{6em}| }{Seu time realiza retrospectiva?} & Total & Comunicação & Relacionamento & Produtividade & Nenhum & Outros \\
      \cline{1-7} \multicolumn{1}{ |m{6em}| }{Sim} & 31 & 16 (51\%) & 3 (9\%) & 17 (54\%) & 6 (19\%) & 3 (9\%) \\
      \cline{1-7} \multicolumn{1}{ |m{6em}| }{Não} & 13 & 8 (61\%) & 4 (30\%) & 6 (46\%) & 1 (7\%) & 3 (23\%) \\
      \cline{1-7} \multicolumn{1}{ |m{6em}| }{Total} & 44 & 24 (54\%) & 7 (15\%) & 23 (52\%) & 7 (15\%) & 6 (13\%) \\
      \cline{1-7}
    \end{tabular}
  \end{adjustwidth}
\end{table}

Os resultados mostram que a retrospectiva ajuda os times ágeis. Entre os que não realizam retrospectiva, o percentual de problemas foi de 61\% para comunicação e 30\% para relacionamento, respectivamente. Já para os times que a utilizam os percentuais foram de 51\% e 9\%. Além disso, aqueles que não usam desta prática aparentam ter problemas mais diversos dos que os que fazem, pois 9\% daqueles que responderam que sim assinalaram a opção outros, enquanto que para aqueles que responderam não o percentual de 23\%.
    
Outro fator que mostra que retrospectivas auxiliam os times a evoluir foram os resultados na opção "Nenhum". Apenas 7\% dos times que não usam esta prática ágil assinalaram esta opção, enquanto que entre os que a utilizam apresentaram um resultado de 19\%. Isso pode ser um indicativo de potenciais problemas no time já que, como visto antes, é suspeito que um time simplesmente não tenha problema algum -- é mais provável que os problemas existam, mas que o time não esteja ciente deles.

Um resultado que contraria a perspectiva inicial é a produtividade. Os times que não realizam retrospectivas apresentaram melhores resultados neste quesito, já que o seu percentual foi de 46\%, contra 54\% dos que fazem a reunião. Este fato pode ocorrer por conta da maior parte dos times que responderam não estão classificados na categoria times distribuídos em \textit{timezones} diversos. Como apresentado acima, estes times são os que mostraram os melhores resultados em termos de produtividade, o que pode ter influenciado o resultado mostrado acima.

Conclui-se que as retrospectivas têm de fato auxiliado os times no processo de melhoria contínua. Porém, ainda há espaço para ajudar as equipes nesta reunião.

Os próximos resultados mostram os principais desafios que os times enfrentam durante a retrospectiva.

\begin{table}[H]
  \begin{adjustwidth}{-4em}{}
    \begin{tabular}{  m{3em} | m{5.5em} | m{5em} | m{5em} | m{5em} | m{5.5em} | m{5em} | m{5em} | }
      \cline{2-8} & \multicolumn{7}{ c| }{Quais desafios seu time enfrenta ao realizar uma retrospectiva?} \\ 
      \cline{1-8} \multicolumn{1}{ |m{3em}| }{Total} & Ultrapassar a duração & Agendar dia, horário e lugar & Discussões de pouco valor & Falta de intimidade entre integrantes & Engajamento das pessoas & Falta de anonimato & Outros \\
      \cline{1-8} \multicolumn{1}{ |m{3em}| }{19} & 10 (52\%) & 1 (5\%) & 8 (42\%) & 4 (21\%) & 9 (47\%) & 1 (5\%) & 3 (15\%) \\
      \cline{1-8}
    \end{tabular}
  \end{adjustwidth}
\end{table}

A pesquisa apontou que os principais desafios enfrentados pelos times ágeis são ultrapassar a duração, falta de engajamento das pessoas e discussões de pouco valor. Uma hipótese levantada por alguns artigos, como Poussard~\cite{poussard},  é que os times sofrem com a falta de engajamento e as discussões de pouco valor por conta deles não variarem as atividades que utilizam nas retrospectivas. Através da pesquisa é possível verificar esta suspeita através da correlação entre os desafios com o fato do time variar ou não as atividades. Os resultados desta análise estão a seguir.

\begin{table}[H]
  \small
  \begin{adjustwidth}{-6.5em}{}
    \begin{tabular}{ m{5.5em} m{3em} | m{5em} | m{5em} | m{5em} | m{5em} | m{5.5em} | m{5em} | m{5em} | }
      \cline{3-9} & & \multicolumn{7}{ c| }{Quais desafios seu time enfrenta ao realizar uma retrospectiva?} \\ 
      \cline{1-9} \multicolumn{1}{ |m{5.5em}| }{Você varia as atividades e/ou formato das suas retrospectivas?} & Total & Ultrapassar a duração & Agendar dia, horário e lugar & Discussões de pouco valor & Falta de intimidade entre integrantes & Engajamento das pessoas & Falta de anonimato & Outros \\
      \cline{1-9} \multicolumn{1}{ |m{5.5em}| }{Sim} & 8 & 4 (50\%) & 0 (0\%) & 3 (37\%) & 1 (12\%) & 4 (50\%) & 1 (12\%) & 2 (25\%) \\
      \cline{1-9} \multicolumn{1}{ |m{5.5em}| }{Não} & 11 & 6 (54\%) & 1 (9\%) & 5 (45\%) & 3 (27\%) & 5 (45\%) & 0 (0\%) & 1 (9\%) \\
      \cline{1-9}
    \end{tabular}
  \end{adjustwidth}
\end{table}

Os resultados mostram que os percentuais dos problemas são próximos entre aqueles que variam ou não as atividades. Porém, alguns deles confirmaram as expectativas, como o fato de que equipes que não variam as retrospectivas têm mais problemas de discussões de pouco valor. Estes times tiveram um percentual de 45\% para este desafio, enquanto para aqueles que variam seus formatos de retrospectiva o resultado foi de 37\%.

Além disso, os times que variam suas atividades aparentam sofrer menos com falta de intimidade entre integrantes em relação às equipes que não variam, pois os resultados foram de 12\% e 27\%, respectivamente.

Para a falta de engajamento das pessoas as porcentagens foram muito próximas e os times que responderam que não variam tiveram um melhor resultado, contrariando as expectativas. Então não ficou evidente que alternar os formatos de retrospectiva ajuda os times com todos os desafios.

Apesar da pesquisa apontar que somente alternar as atividades não é suficiente, incentivar os times a variar os formatos de retrospectiva que eles utilizam pode ajudá-los a enfrentar alguns dos desafios desta reunião.

Uma outra forma conhecida para diminuir os problemas em uma retrospectiva é o uso do facilitador. A correlação entre a quantidade de desafios com o uso ou não do facilitador é apresentada na próxima tabela.

\begin{table}[H]
  \small
  \begin{adjustwidth}{-6.5em}{}
    \begin{tabular}{ m{5.5em} m{3em} | m{5em} | m{5em} | m{5em} | m{5em} | m{5.5em} | m{5em} | m{5em} | }
      \cline{3-9} & & \multicolumn{7}{c|}{Quais desafios seu time enfrenta ao realizar uma retrospectiva?} \\ 
      \cline{1-9} \multicolumn{1}{ |m{5.5em}| }{Alguém fica responsável por facilitar as retrospectivas?} & Total & Ultrapassar a duração & Agendar dia, horário e lugar & Discussões de pouco valor & Falta de intimidade entre integrantes & Engajamento das pessoas & Falta de anonimato & Outros \\
      \cline{1-9} \multicolumn{1}{ |m{5.5em}| }{Sim} & 17 & 8 (47\%) & 1 (5\%) & 8 (47\%) & 4 (23\%) & 9 (52\%) & 1 (5\%) & 3 (17\%) \\
      \cline{1-9} \multicolumn{1}{ |m{5.5em}| }{Não} & 2 & 2 (100\%) & 0 (0\%) & 0 (0\%) & 0 (0\%) & 0 (0\%) & 0 (0\%) & 0 (0\%) \\
      \cline{1-9}
    \end{tabular}
  \end{adjustwidth}
\end{table}

O questionário mostrou que é quase unânime o uso do facilitador nas retrospectivas. São 17 dos 19 times que responderam a pesquisa que usam deste recurso, o que corresponde a 89\% das equipes.
    
Como apenas dois times responderam que não têm um facilitador responsável por suas reuniões, não é possível encontrar padrões de desvio entre o grupo maior e esse par de respostas. Porém, nota-se que ambas as equipes que responderam ''Não'' reportaram problemas em ultrapassar a duração da reunião -- é possível que o uso desta técnica ajude no melhor controle do tempo.

Mesmo usando o facilitador, aproximadamente metade dos times tem problemas como ultrapassar a duração (47\%), discussões de pouco valor (47\%) e falta de engajamento das pessoas (52\%). Logo, somente com uma pessoa facilitando a reunião não garante que as retrospectivas serão bem-feitas.

\subsubsection*{Resultados das perguntas qualitativas}

Para complementar as informações obtidas com as respostas quantitativas, a pesquisa também possui questões qualitativas que se relacionam com os dados anteriores. 

Uma das perguntas busca investigar o porquê do baixo número de equipes geograficamente distribuídas fazendo retrospectiva. Como os resultados anteriores demonstram problemas de sincronização nestes times, um motivo para eles não realizarem retrospectivas é devido à impossibilidade de reunir todos os membros para a discussão em algum horário. Além disso, alguns artigos encontrados pressupõem que somente os gerentes e líderes de projeto participavam das discussões e decidiam o que seria feito.

Duas respostas apontaram que a razão para o time não realizar a retrospectiva é devido a falta de tempo, já que ambos alegaram que a equipe está passando por período de alta demanda. Outros dois participantes responderam que o motivo vem a partir de uma decisão dos gestores do projeto. Uma dessas respostas também revelou que o gerente não possui experiência com metodologias ágeis. Além disso, um outro participante também indicou que não é cultura da empresa utilizar desta prática. 

Somente com estas informações não é possível tirar conclusões definitivas sobre o porquê deles não realizarem retrospectivas. Uma hipótese é que esses times ainda não possuem experiência suficiente com métodos ágeis e também não dispõem de pessoas engajadas em agilidade que queiram aplicar retrospectivas como forma de inspecionar problemas. Logo, eles acabam não vendo o valor dessa reunião e o retorno que ela proporciona. Essa hipótese é reforçada a partir das respostas sobre como os problemas nesses times são discutidos e resolvidos, quem participa das discussões e quem aplica as mudanças para solucionar o problema.

Os times que responderam anteriormente que não fazem retrospectivas devido a alta demanda indicaram que realizam \textit{conference calls} entre os membros do time para debater sobre os problemas do time. Em geral, essas reuniões não tem um horário previsto para acontecer e todos os integrantes participam. Percebe-se, assim, uma inconsistência nas respostas destes participantes, pois ambos alegam que o time não possui tempo para realizar a retrospectiva, mas eles conseguem mobilizar todos os membros para uma reunião onde serão discutidos os problemas. Desta forma, torna-se mais provável que a razão pela qual eles não utilizam a retrospectiva é pela falta de experiência com agilidade, não conhecer a fundo esta prática e saber o valor que esta reunião pode proporcionar.

Da mesma forma, também houve inconsistências nas respostas dos outros times que apontaram que a razão para não realizarem a reunião é por conta de uma decisão dos gestores ou por não ser parte da cultura da empresa. Estes times conversam via chat para discutir os seus problemas e todos os membros que possuem disponibilidade participam ativamente do debate. Caso as pessoas encontrem uma solução interessante para o problema, ela é informada para o restante da equipe e para o gerente do projeto. 

Logo, a forma como esses times promovem melhoria contínua é muito semelhante à retrospectiva. Provavelmente a única diferença é que algumas das atividades utilizadas em retrospectivas motivam o time a conversar sobre o futuro. Nas respostas que os participantes deram quando questionados de que forma o time identifica os problemas futuros e trabalha para evitá-los, a grande maioria indicou que a equipe não tem esse tipo de preocupação. Apenas um participante comentou que os integrantes passam por essa discussão, mas focada apenas no código da aplicação. Ou seja, eles não possuem uma preocupação em debater sobre o futuro do projeto e do time.

Como nenhum dos participantes mencionou o nome de alguma prática de melhoria contínua, como o \textit{Kaizen} ou \textit{Hansei}, não é possível concluir se estas reuniões são baseadas em algum outro tipo de prática ágil diferente da retrospectiva. Além disso, nota-se que as decisões não são tomadas somente pelos gerentes, \textit{coachs} e líderes de projeto, mas sim por todos os integrantes da equipe. Desta forma, não existem grandes impeditivos para mesmo os times distribuídos realizarem suas reuniões seguindo os preceitos da retrospectiva. Um sistema que auxilie essas equipes a sincronizarem seus horários e forneça uma plataforma para realizar essas reuniões estruturadas pode motivar os times a usarem a retrospectiva como prática de melhoria contínua.

\subsection{Conclusão}

A pesquisa forneceu dados importantes que não eram encontrados em publicações na comunidade. Por exemplo, as informações de quais estruturas de times usam a retrospectiva, que dificuldades estas equipes enfrentam na reunião e que outras formas de melhoria contínua são adotadas por aqueles que não fazem a retrospectiva.

Sem a pesquisa, não é possível ter uma ideia da realidade atual que os diversos times ágeis estão enfrentando e que ainda há muitas dificuldades para serem superadas. 

Saber as reais necessidades dos times ágeis é vital para planejar uma estratégia de como auxiliá-los. Pensando nisso, foi possível planejar uma aplicação focada em melhoria contínua, usando como base os resultados obtidos a partir da pesquisa.