\section{Pesquisa}

Tanto a melhoria contínua quanto a retrospectiva já são conceitos muito conhecidos e utilizados pela comunidade ágil. Como cada time possui a suas próprias necessidades e dificuldades, é comum a equipe adaptar estes conceitos para a sua situação. Isto dificulta para encontrar informações sobre como a melhoria contínua é aplicada atualmente e saber quais são as dificuldades que os times ágeis enfrentam ao realizar estas reuniões, principalmente a retrospectiva. 

Alguns destes dados só foram obtidos a partir de uma revisão sistemática da literatura sobre retrospectivas e quais são os problemas em seu desenvolvimento. No entanto, a maior parte do material que existe hoje foca na aplicação da reunião e nos passos durante a retrospectiva.

Os artigos que mostram o estado atual dos times ágeis consistem em relatórios de questionários que indicam a quantidade de times que fazem retrospectivas, como as pesquisas da VersionOne~\cite{versionOne} e da Scrum Alliance~\cite{scrumAlliance}. Em geral, os relatórios mostram a porcentagem de times que estão em cada etapa da adoção dos métodos ágeis, desde a introdução dessas metodologias na equipe até sua total adequação. Eles também mostram dados como a distribuição demográfica dos times que adotaram agilidade e quais métricas as equipes utilizam.

Boa parte das pesquisas apontavam que o uso de retrospectivas vem constantemente aumentando nos últimos anos. No entanto, são poucos os trabalhos que apresentam as dificuldades enfrentadas pelos times e, em geral, o assunto é tratado apenas para equipes locais. Existem duas possibilidades para essa falta de informação sobre equipes remotas: times usarem outras formas de promover melhoria contínua ou não haver publicações sobre o tema na comunidade.

%Está ruim! (em cima novo, comentário a baixo como estava)
Devido a falta de material de apoio para levantar estas informações, foi feito uma pesquisa sobre as práticas utilizadas por times locais e distribuídos, e quais desafios eles enfrentavam.

%Na falta de material de apoio para levantar os requisitos da aplicação, ou autores e o orientador decidiram que era necessário realizar uma pesquisa. Assim, foi desenvolvido um questionário que buscava entender quais eram as práticas utilizadas por times locais e distribuídos, e quais desafios eles enfrentavam.

%Está ruim! (em cima novo, comentário a baixo como estava)
Para abranger uma parte ainda maior da comunidade ágil e enviesar menos o questionário em relação à ferramenta usada para trazer o princípio da melhoria contínua à tona, o assunto da pesquisa seria não apenas retrospectivas, mas o tópico maior: como times ágeis promovem melhoria continuamente.

%O próprio assunto a ser abordado na pesquisa foi alterado, nessa fase. Nesse momento, para abranger uma parte ainda maior da comunidade ágil e enviesar menos o questionário em relação à ferramenta usada para trazer o princípio da melhoria contínua à tona, decidiu-se que o assunto da pesquisa seria não apenas retrospectivas, mas o tópico maior: como times ágeis promovem melhoria continuamente.

\subsection{Motivação e questões da pesquisa}

Considerando, então, que o tema da pesquisa é de melhoria contínua, as questões deveriam ser imparciais de modo a não direcionar as respostas em torno das retrospectivas. Como retrospectivas são as formas mais frequentemente indicadas para times ágeis promoverem melhoria contínua, seria interessante ter questões especificas para os time que realizam retrospectivas, de modo a entender como as executam.

Também não achamos que seria possível desenvolver uma pesquisa em que todas os formatos de equipes ágeis fossem atingidos. Assim, a dupla decidiu focar em duas frentes: equipes que fazem retrospectivas e equipes que não as fazem. Dessa forma, seria possível investigar mais isoladamente os problemas específicos desses contextos e como os times os resolvem.

Apesar de ser duas frentes diferentes o enfoque era o mesmo. Assim a pesquisa deveria ter uma parte inicial que focasse nos problemas de equipes em geral e na segunda parte particularidades sobre como cada equipe busca solucionar os problemas.

A pesquisa envolve entender como times ágeis buscam melhoria contínua. A retrospectiva é utilizada por diversos métodos ágeis e muitos times, mas existem diversos desafios para se realizar esta reunião. A ideia era encontrar os principais problemas que as equipes enfrentam e a relação deles com a estrutura de cada time. Além disso, havia também uma forte hipótese de que, embora nem todos os times façam retrospectivas, eles ainda adotem alguma outra forma de promover melhoria contínua. 

Para times que realizam retrospectivas a intenção foi entender o processo para realização da retrospectiva e os desafios encontrados. Como existem formas sugeridas para a realização dessas reuniões e a dupla de autores trabalha em um ambiente com equipes ágeis que realizam retrospectiva, alguns problemas já eram esperados e foram colocados na pesquisa. Porém, como a estrutura da retrospectiva nem sempre é seguida, poderiam existir outros problemas a serem analisados e levados em consideração no desenvolvimento.

Para times que não realizam retrospectivas existem incertezas e dúvidas sobre a não realização das retrospectivas. Primeiro, gostaríamos de entender por que não utilizavam a prática, se já a haviam utilizado e encontraram problemas, ou se não gostavam da estrutura. Segundo, queríamos entender como faziam para promover a melhoria contínua, se havia alguma outra forma de reunião ou se não faziam uso de reuniões para esse efeito.

\subsection{Resultados}

O tempo para a coleta de respostas foi de 1 mês e foram obtidas 44 respostas. A pesquisa era composta de respostas quantitativas, com o intuito de coletar os problemas enfrentados e a utilização de retrospectivas, e qualitativas, que buscavam entender como os times resolviam seus problemas.

\subsubsection*{Resultados das perguntas quantitativas}

A primeira informação estudada foram os problemas que cada uma das estruturas de times ágeis enfrentam. Foi feita a separação em:

\begin{itemize}
  \item Local: time que estão no mesmo ambiente
  \item Distribuído no mesmo timezone: times que não estão no mesmo ambiente mas no mesmo fuso horário.
  \item Distribuído em timezones diversos: times que não estão no mesmo ambiente e estão distribuídos em diferentes fusos horários.
\end{itemize}



Os resultados seguem na tabela abaixo.

\begin{table}[H]
  \begin{adjustwidth}{-3em}{}
    \begin{tabular}{  m{6em}  m{3em} | m{6em} | m{7em} | m{6em} | m{5em} | m{5em} | }
      \cline{3-7} & & \multicolumn{5}{ c| }{Quais problemas seu time enfrenta? } \\ 
      \cline{1-7} \multicolumn{1}{ |m{6em}| }{Estrutura majoritária do time} & Total & Comunicação & Relacionamento & Produtividade & Nenhum & Outros \\
      \cline{1-7} \multicolumn{1}{ |m{6em}| }{Local} & 27 & 15 (55,56\%) & 4 (14,81\%) & 18 (66,67\%) & 4 (14,81\%) & 2 (7,41\%) \\ 
      \cline{1-7} \multicolumn{1}{ |m{6em}| }{Distribuído no mesmo timezone} & 11 & 5 (45,45\%) & 1 (9,09\%) & 4 (36,36\%) & 3 (27,27\%) & 1 (9,09\%) \\
      \cline{1-7} \multicolumn{1}{ |m{6em}| }{Distribuído em timezones diversos} & 6 & 4 (66,67\%) & 2 (33,33\%) & 1 (16,67\%) & 0 (0\%) & 3 (50\%) \\
      \cline{1-7} \multicolumn{1}{ |m{6em}| }{Total} & 44 & 24 (54,54\%) & 7 (15,91\%) & 23 (52,27\%) & 7 (15,91\%) & 6 (13,64\%) \\
      \cline{1-7}
    \end{tabular}
  \end{adjustwidth}
\end{table}

Nota-se que o desafio mais enfrentado por times ágeis é a comunicação. A pesquisa mostra que aproximadamente 54,54\% das equipes passam por esse problema. Além disso, os times distribuídos em timezones diversos tendem a sofrer mais com este problema, já que 66,67\% deles passam por esta dificuldade, enquanto que os percentuais dos locais e distribuídos no mesmo timezone foram de 55,56\% e 45,45\%, respectivamente.
  
Outro desafio no qual os times distribuídos em diversos fusos horários aparentam ter mais dificuldades do que as outras equipes é a de relacionamento. Aproximadamente 33,33\% dessas equipes relataram que passam por esse problema, contra apenas 14,81\% dos locais e 9,09\% dos distribuídos no mesmo timezone.
  
No entanto, apesar dos times distribuídos em diversos fusos terem mais problemas de comunicação e relacionamentos, eles aparentemente possuem o melhor rendimento entre as três estruturas. Apenas 16,67\% deles possuem problemas de produtividade. Já para os times distribuídos no mesmo timezone o valor chegou a 36,36\%, ou seja, mais do que o dobro. Porém, os times que mais sofrem com este problema são os locais que apresentaram um percentual de 66.67\%.
  
Por fim, um resultado que chamou a atenção foi um outro problema apresentado pelos times distribuídos em timezones diversos. A pesquisa mostrou que eles tem dificuldade para sincronizar todos os seus membros. Como as pessoas trabalham em fusos horários diferentes fica difícil definir as datas para os diversos tipos de reuniões ágeis, como o \textit{daily}, o \textit{planning}, a \textit{review} e a retrospectiva.

Depois de analisar quais são os desafios que cada um dos tipos de times ágeis enfrentam, a próxima etapa foi verificar como eles tentam resolver os seus problemas. Diversas pesquisas e artigos da área já mostram que a grande maioria das equipes adotam a retrospectiva como uma das práticas ágeis utilizadas. No entanto, com a pesquisa foi passível verificar como esta reunião era aceita por cada um dos diversos tipos de equipes.

\begin{table}[H]
  \centering
  \begin{tabular}{  m{6em}  m{3em} | m{6em} | m{7em} |}
    \cline{3-4} & & \multicolumn{2}{ c| }{Seu time realiza retrospectivas?} \\ 
    \cline{1-4} \multicolumn{1}{ |m{6em}| }{Estrutura majoritária do time} & Total & Sim & Não \\
    \cline{1-4} \multicolumn{1}{ |m{6em}| }{Local} & 27 & 21 (77,78\%) & 6 (22,22\%) \\
    \cline{1-4} \multicolumn{1}{ |m{6em}| }{Distribuído no mesmo timezone} & 11 & 9 (81,82\%) & 2 (18,18\%) \\
    \cline{1-4} \multicolumn{1}{ |m{6em}| }{Distribuído em timezones diversos} & 6 & 1 (16,67\%) & 5 (83,33\%) \\
    \cline{1-4} \multicolumn{1}{ |m{6em}| }{Total} & 44 & 31 (70,45\%) & 13 (29,55\%) \\
    \cline{1-4}
  \end{tabular}
\end{table}

Pode-se perceber que o resultado geral da pesquisa foi condizente com o apresentado por outros artigos publicados, já que 70,45\% dos participantes responderam que realizam retrospectivas. No entanto, é interessante notar como foram as respostas para cada uma das estruturas das equipes. Claramente os times locais e distribuídos no mesmo timezone são os que mais utilizam desta reunião como prática de melhoria contínua. Em contrapartida, aqueles que são distribuídos em timezones diversos não costumam realizar retrospectivas já que apenas 16,67\% responderam que sim.

Ainda era necessário verificar se a retrospectiva de fato auxilia os times a promover melhoria contínua. Para isso, foi necessário comparar a quantidade de problemas entre equipes que realizam ou não a reunião. Os resultados obtidos encontram-se na tabela a seguir.

\begin{table}[H]
  \begin{adjustwidth}{-3em}{}
    \begin{tabular}{ m{6em} m{3em} | m{6em} | m{7em} | m{6em} | m{5em} | m{5em} | }
      \cline{3-7} & & \multicolumn{5}{ c| }{Quais problemas seu time enfrenta?} \\ 
      \cline{1-7} \multicolumn{1}{ |m{6em}| }{Seu time realiza retrospectiva?} & Total & Comunicação & Relacionamento & Produtividade & Nenhum & Outros \\
      \cline{1-7} \multicolumn{1}{ |m{6em}| }{Sim} & 31 & 16 (51,61\%) & 3 (9,68\%) & 17 (54,84\%) & 6 (19,35\%) & 3 (9,68\%) \\
      \cline{1-7} \multicolumn{1}{ |m{6em}| }{Não} & 13 & 8 (61,53\%) & 4 (30,77\%) & 6 (46,15\%) & 1 (7,69\%) & 3 (23,08\%) \\
      \cline{1-7} \multicolumn{1}{ |m{6em}| }{Total} & 44 & 24 (54,54\%) & 7 (15,91\%) & 23 (52,27\%) & 7 (15,91\%) & 6 (13,64\%) \\
      \cline{1-7}
    \end{tabular}
  \end{adjustwidth}
\end{table}

Analisando os resultados foi possível notar que a retrospectiva parece ajudar os times ágeis. Entre os que não realizam retrospectiva, o percentual de problemas foi de 61,53\% para comunicação e 30,77\% para relacionamento, respectivamente. Já para os times que a utilizam os percentuais foram de 51,61\% e 9,68\%. Além disso, aqueles que não usam desta prática aparentam ter problemas mais diversos dos que os que fazem, pois 9,68\% daqueles que responderam que sim assinalaram a opção outros, enquanto que para aqueles que responderam não o percentual de 23,68\%. Note que, para essa análise, os dados de todas as categorias de times estão juntos.
    
Outro fator que mostra que retrospectivas auxiliam os times a evoluir foram os resultados na opção "Nenhum". Apenas 7,69\% dos times que não usam esta prática ágil assinalaram esta opção, enquanto que entre os que a utilizam apresentaram um resultado de 19,35\%. Isso pode ser interpretado como um indicativo de potenciais problemas no time já que, como visto antes, é bastante suspeito que um time simplesmente não tenha problema algum -- é mais provável que os problemas existam, mas que o time não esteja ciente deles.

No entanto, um resultado que contraria a perspectiva inicial é a produtividade. Os times que não realizam retrospectivas apresentaram melhores resultados neste quesito, já que o seu percentual foi de 46,15\%, contra 54,84\% dos que fazem a reunião. Uma das hipóteses da dupla para este fato ter ocorrido é que a maior parte dos times que responderam não estão classificados na categoria times distribuídos em timezones diversos. Como apresentado acima, estes times são os que mostraram os melhores resultados em termos de produtividade, o que pode ter influenciado o resultado mostrado acima.

Portanto, concluiu-se que as retrospectivas têm de fato auxiliado os times no processo de melhoria contínua. Porém, ainda há espaço para ajudar estas equipes nesta reunião, já que muitos deles ainda possuem os mesmos problemas. Desta forma, construir uma aplicação que auxilie neste processo se mostrou cada vez mais factível.

Com as análises sobre a viabilidade de uma aplicação, precisava-se avaliar quais eram os principais desafios que os times enfrentam durante a retrospectiva.

\begin{table}[H]
  \begin{adjustwidth}{-4em}{}
    \begin{tabular}{  m{3em} | m{5.5em} | m{5em} | m{5em} | m{5em} | m{5.5em} | m{5em} | m{5em} | }
      \cline{2-8} & \multicolumn{7}{ c| }{Quais desafios seu time enfrenta ao realizar uma retrospectiva?} \\ 
      \cline{1-8} \multicolumn{1}{ |m{3em}| }{Total} & Ultrapassar a duração & Agendar dia, horário e lugar & Discussões de pouco valor & Falta de intimidade entre integrantes & Engajamento das pessoas & Falta de anonimato & Outros \\
      \cline{1-8} \multicolumn{1}{ |m{3em}| }{19} & 10 (52,63\%) & 1 (5,26\%) & 8 (42,11\%) & 4 (21,05\%) & 9 (47,37\%) & 1 (5,26\%) & 3 (15,79\%) \\
      \cline{1-8}
    \end{tabular}
  \end{adjustwidth}
\end{table}

A pesquisa apontou que os principais desafios enfrentados pelos times ágeis são ultrapassar a duração, falta de engajamento das pessoas e discussões de pouco valor. Assim, durante o levantamento dos requisitos esses três pontos teriam que ser levados em consideração. 

Uma das hipóteses iniciais para times sofrerem com a falta de engajamento e as discussões de pouco valor era por conta deles não variarem as atividades que utilizavam nas retrospectivas. Para verificar esta suspeita, foi preciso correlacionar os desafios com o fato do time variar ou não as atividades. Os resultados desta análise estão a seguir.

\begin{table}[H]
  \small
  \begin{adjustwidth}{-6.5em}{}
    \begin{tabular}{ m{5.5em} m{3em} | m{5em} | m{5em} | m{5em} | m{5em} | m{5.5em} | m{5em} | m{5em} | }
      \cline{3-9} & & \multicolumn{7}{ c| }{Quais desafios seu time enfrenta ao realizar uma retrospectiva?} \\ 
      \cline{1-9} \multicolumn{1}{ |m{5.5em}| }{Você varia as atividades e/ou formato das suas retrospectivas?} & Total & Ultrapassar a duração & Agendar dia, horário e lugar & Discussões de pouco valor & Falta de intimidade entre integrantes & Engajamento das pessoas & Falta de anonimato & Outros \\
      \cline{1-9} \multicolumn{1}{ |m{5.5em}| }{Sim} & 8 & 4 (50\%) & 0 (0\%) & 3 (37,5\%) & 1 (12,5\%) & 4 (50\%) & 1 (12,5\%) & 2 (25\%) \\
      \cline{1-9} \multicolumn{1}{ |m{5.5em}| }{Não} & 11 & 6 (54,54\%) & 1 (9,09\%) & 5 (45,45\%) & 3 (27,27\%) & 5 (45,45\%) & 0 (0\%) & 1 (9,09\%) \\
      \cline{1-9}
    \end{tabular}
  \end{adjustwidth}
\end{table}

Analisando os resultados, é possível notar que eles foram próximos entre aqueles que variam ou não as atividades. Porém, alguns deles confirmaram as expectativas, como o fato de que equipes que não variam as retrospectivas têm mais problemas de discussões de pouco valor. Estes times tiveram um percentual de 45,45\% para este desafio, enquanto para aqueles que variam seus formatos de retrospectiva o resultado foi de 37,5\%.

Além disso, é interessante notar que times que variam suas atividades aparentam sofrer menos com falta de intimidade entre integrantes em relação às equipes que não variam, pois os resultados foram de 12,5\% e 27,27\%, respectivamente.

No entanto, contrariando as expectativas, não ficou evidente que alternar os formatos de retrospectiva ajuda os times na falta de engajamento das pessoas. Para este desafio as porcentagens foram muito próximas e os times que responderam que não variam tiveram um melhor resultado.

Desta forma, incentivar os times a variar os formatos de retrospectiva que eles utilizam pode ajudá-los a enfrentar alguns dos desafios desta reunião. No entanto, a pesquisa apontou que somente alternar as atividades não será suficiente. Logo, durante o levantamento dos requisitos da aplicação foi necessário pensar em outras estratégias de como auxiliar as equipes além de incentivá-los a variar suas atividades.

Uma das formas conhecidas de diminuir os problemas em uma retrospectiva é o uso do facilitador. Para verificar se esta estratégia realmente ajuda os times, a dupla analisou a correlação entre a quantidade de desafios com o uso ou não do facilitador.


\begin{table}[H]
  \small
  \begin{adjustwidth}{-6.5em}{}
    \begin{tabular}{ m{5.5em} m{3em} | m{5em} | m{5em} | m{5em} | m{5em} | m{5.5em} | m{5em} | m{5em} | }
      \cline{3-9} & & \multicolumn{7}{c|}{Quais desafios seu time enfrenta ao realizar uma retrospectiva?} \\ 
      \cline{1-9} \multicolumn{1}{ |m{5.5em}| }{Alguém fica responsável por facilitar as retrospectivas?} & Total & Ultrapassar a duração & Agendar dia, horário e lugar & Discussões de pouco valor & Falta de intimidade entre integrantes & Engajamento das pessoas & Falta de anonimato & Outros \\
      \cline{1-9} \multicolumn{1}{ |m{5.5em}| }{Sim} & 17 & 8 (47,06\%) & 1 (5,88\%) & 8 (47,06\%) & 4 (23,53\%) & 9 (52,94\%) & 1 (5,88\%) & 3 (17,64\%) \\
      \cline{1-9} \multicolumn{1}{ |m{5.5em}| }{Não} & 2 & 2 (100\%) & 0 (0\%) & 0 (0\%) & 0 (0\%) & 0 (0\%) & 0 (0\%) & 0 (0\%) \\
      \cline{1-9}
    \end{tabular}
  \end{adjustwidth}
\end{table}

O questionário mostrou que é quase unânime o uso do facilitador nas retrospectivas. São 17 dos 19 times que responderam a pesquisa que usam deste recurso, o que corresponde a 89,47\% das equipes.
    
Como apenas dois times responderam que não têm um facilitador responsável por suas reuniões, foi inviável encontrar padrões de desvio entre o grupo maior e esse par de respostas. Foi interessante, porém, notar que as duas equipes que responderam que "Não" reportaram problemas em ultrapassar a duração da reunião -- é possível que o uso desta técnica ajude no melhor controle do tempo da reunião.

Um fato que chamou a atenção foi que mesmo usando o facilitador, aproximadamente metade dos times tem problemas como ultrapassar a duração (47,06\%), discussões de pouco valor (47,06\%) e falta de engajamento das pessoas (52,94\%). Logo, somente com uma pessoa facilitando a reunião não garante que as retrospectivas serão bem-feitas.

\subsubsection*{Resultados das perguntas qualitativas}

Depois de analisar as respostas quantitativas, o próximo passo foi relacionar os resultados com as perguntas qualitativas para averiguar questões que ficaram em aberto. A principal delas era investigar o porquê do baixo número de equipes geograficamente distribuídas fazendo retrospectiva.
  
Desta forma, foi necessário investigar porque equipes geograficamente distribuídas apresentaram um percentual tão baixo. Devido os problemas para sincronização relatados por esses times, a suspeita era que eles não realizavam retrospectivas devido à impossibilidade de reunir todos os membros para a discussão em algum horário. Além disso, com base nos artigos lidos previamente, pressupôs-se que somente os gerentes e líderes de projeto participavam dessas discussões e decidiam o que seria feito.

Duas respostas apontaram que a razão para o time não realizar a retrospectiva é devido a falta de tempo, já que ambos alegaram que a equipe está passando por período de alta demanda. Outros dois participantes responderam que o motivo vem a partir de uma decisão dos gestores do projeto. Uma dessas respostas também revelou que o gerente não possui experiência com metodologias ágeis. Além disso, um outro participante também indicou que não é cultura da empresa utilizar desta prática. Sem um contato direto com os líderes do projeto não nos sentimos confortáveis em tirar conclusões definitivas sobre o porquê deles não realizarem retrospectivas. A hipótese é que esses times ainda não possuem experiência suficiente com métodos ágeis e também não dispõem de pessoas engajadas em agilidade que queiram aplicar retrospectivas como forma de inspecionar problemas. Logo, eles acabam não vendo o valor dessa reunião e o retorno que ela proporciona. Essa hipótese foi reforçada quando se analisou as respostas de como os problemas nesses times são discutidos e resolvidos, quem participa dessas discussões e quem aplica as mudanças para solucionar o problema.

Os times que responderam anteriormente que não fazem retrospectivas devido a alta demanda indicaram que realizam \textit{conference calls} entre os membros do time para debater sobre os problemas do time. Em geral, essas reuniões não tem um horário previsto para acontecer e todos os integrantes participam. Percebe-se, assim, uma inconsistência nas respostas destes participantes, pois ambos alegam que o time não possui tempo para realizar a retrospectiva, mas eles conseguem mobilizar todos os membros para uma reunião onde serão discutidos os problemas. Desta forma, ficou mais forte a suspeita de que a real razão pela qual eles não utilizam a retrospectiva é pela falta de experiência com agilidade, não conhecer a fundo esta prática e saber o valor que esta reunião pode proporcionar.

Da mesma forma, também houve inconsistências nas respostas dos outros times que apontaram que a razão para não realizarem a reunião é por conta de uma decisão dos gestores ou por não ser parte da cultura da empresa. Estes times conversam via chat para discutir os seus problemas e todos os membros que possuem disponibilidade participam ativamente do debate. Caso as pessoas encontrem uma solução interessante para o problema, ela é informada para o restante da equipe e para o gerente do projeto. 

Logo, a forma como esses times também promovem melhoria contínua é muito semelhante à retrospectiva. Provavelmente a única diferença é que algumas das atividades utilizadas em retrospectivas motivam o time a conversar sobre o futuro. Nas respostas que os participantes deram quando questionados de que forma o time identifica os problemas futuros e trabalha para evitá-los, a grande maioria indicou que a equipe não tem esse tipo de preocupação. Apenas um participante comentou que os integrantes passam por essa discussão, mas focada apenas no código da aplicação. Ou seja, eles não possuem uma preocupação em debater sobre o futuro do projeto e do time.

Como nenhum dos participantes mencionou o nome de alguma prática de melhoria contínua, como o \textit{Kaizen} ou \textit{Hansei}, não foi possível concluir se estas reuniões são baseadas em algum outro tipo de prática ágil diferente da retrospectiva. Além disso, contrariando a expectativa inicial, as decisões não são tomadas somente pelos gerentes, \textit{coachs} e líderes de projeto, mas sim por todos os integrantes da equipe. Desta forma, foi possível concluir que não existem grandes impeditivos para mesmo os times distribuídos realizarem suas reuniões seguindo os preceitos da retrospectiva. Um sistema que auxilie essas equipes a sincronizarem seus horários e forneça uma plataforma para realizar essas reuniões estruturadas pode motivar os times a usarem a retrospectiva como prática de melhoria contínua.

\subsection{Conclusão}

A partir das análises da pesquisa, obteve-se dados importantes que não eram encontrados em publicações na comunidade. Por exemplo, as informações de quais estruturas de times usam a retrospectiva, que dificuldades estas equipes enfrentam na reunião e que outras formas de melhoria contínua são adotadas por aqueles que não fazem a retrospectiva.

Sem a pesquisa, não seria possível perceber a realidade atual que os diversos times ágeis estão enfrentando e que ainda há muitas dificuldades para serem superadas. 
%TODO : revisar esta frase com a Ceci
Saber as suas reais necessidades foi vital para pensar como este trabalho poderia auxiliar estas pessoas. Pensando nisso, a partir dos resultados teve início o planejamento de uma aplicação focada em melhoria contínua e com base nas análises feitas a partir da pesquisa.