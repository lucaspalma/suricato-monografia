\section{Conclusão}

A idéia inicial do trabalho de desenvolver um aplicação envolvendo diversas tecnologias e seu aprendizado durante o projeto foram plenamente atingidos. Mas as expectativas iniciais do aprendizado eram menores do que o que foi atingido.

No ínicio do projeto a proposta do prof. Dr. Alfredo Goldman era uma revisão sistemática sobre retrospectiva proporcionou uma experiência diferente. Através dessa revisão, o conhecimento da equipe sobre metódos agéis e em especial sobre melhoria contínua e retrospectiva tornou-se mais profundo.

A frustação de não encontrar artigos e pesquisas mais detalhados no assunto foi uma situação inesperada. O desenvolvimento de uma pesquisa foi algo novo e desafiador no trabalho. Nenhum dos autores tinha experiência em fazer uma pesquisa acadêmica, o que tornou o processo em um constante aprendizado. Se não fosse a ajuda e influência da Cecília nas comunidades agéis não seria possível conseguir um conjunto de respostas relevantes.

Outra dificuldade encontrada foi analisar os dados coletados. Os membros não sabiam como interpretar e correlacionar as respostas. A equipe recebeu a ajuda de pessoas da Caelum, em especial ao Mauricio Aniche, e conseguiu encontrar uma forma de analisar os dados, relacionar todas as respostas, e chegar a algumas conclusões. Estas foram vitais na hora de planejar os requisitos da aplicação.

Já havia uma idéia das tecnologias e ferramentas a se utiizar. Os autores já eram familiarizados com Java e Hibernate, então o desafio foi aprender a utilizar e resolver os problemas com as outras tecnologias incorporadas. O principal desafio foi em relação aos WebSockets e SpringSecurity, mas com a ajuda do senhor Alberto Tavares foi possível ajustar toda a configuração do sistema. 

Para o escopo do trabalho o objetivo de colocar a aplicação no ar foi atingido. Como o projeto é open source, a equipe agora pode abrir para contribuições externas, já que apenas alguns requisitos foram implementados e ainda há espaço para evoluir.  
O próximo passo do projeto é continuar o desenvolvimento dos requisitos gerados da pesquisa e planejados pela equipe. Por exemplo, um relógio marcando o tempo da reunião, avisos via email e no próprio site sobre as próximas reuniões agendadas, sistema de sugestão de atividades baseado no histórico do time e  usuários criarem novas atividades.

No entanto, o sistema já possui uma quantidade razoável de funcionalidades e consegue ser usado para reuniões de melhoria contínua. Assim, os idealizadores do projeto acreditam que neste momento é mais interessante segurar a implementação de novos recursos e começar a divulgar o sistema para ser usado por mais times além dos presentes na Caelum.

A ideia é primeiro divulgar o sistema para os times que responderam a pesquisa e mandaram qual o email para contato. Se esses times começarem a utilizar o Suricato é possível refazer a pesquisa e averiguar se o Suricato ajudou no seu processo de melhoria contínua.

Além disso, pretende-se também divulgar o sistema em diversas listas da comunidade ágil e verificar se o Suricato consegue ter uma boa aceitação. Também será disponibilizado no site uma área para as pessoas mandarem sugestões com melhorias que podem ser feitas no sistema ou novas funcionalidades.

Futuramente, ficou planejado realizar uma nova pesquisa buscando responder algumas perguntas que ficaram em aberto, como:

\begin{enumerate}
	\item Times passam pro problemas de falta de engajamento porque não variam os tipos de retrospectiva?
	\item Como exatamente os times distribuídos fazem as suas reuniões de melhoria contínua? Que ferramentas eles utilizam?
	\item A figura do facilitador pode ajudar com algumas das dificuldades encontradas em retrospectivas?
	\item Retrospectiva realmente não tem auxiliado os times com problemas de produtividade? Ou isso ocorre por conta dos times distribuídos serem realmente mais produtivos e terem puxado para  o percentual de times que tem esse problema e que não fazem retrospectiva?
\end{enumerate}

Por fim, caso o Suricato se prove uma ferramenta útil na comunidade, será possível escrever alguns artigos para revistas e submeter palestras em eventos ágeis comentando sobre a ferramenta e os resultados obtidos. 