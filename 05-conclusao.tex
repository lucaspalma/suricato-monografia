\section{Conclusão}

Diversos artigos e pesquisas na comunidade indicam que um número crescente de times utiliza a prática de retrospectivas para promover melhoria contínua em seus ambientes. A pesquisa promovida por este trabalho corrobora com essas informações.

Nota-se que times ágeis apresentam uma grande consciência sobre problemas de comunicação, relacionamento e produtividade, embora a estrutura majoritária do time -- isto é, se o trabalho é feito localmente, distribuído no mesmo fuso horário, ou distribuído em diversos fusos horários -- influencie bastante nas questões enfrentadas pelos times pesquisados.

A partir dos problemas encontrados na pesquisa, uma aplicação foi desenvolvida para auxiliar os times. Contudo, um grande valor gerado pelo trabalho são os dados a respeito do estado atual da retrospectiva e da melhoria contínua na comunidade. Este tipo de informação ainda não é encontrado com frequência em publicações da área.

Para o escopo do trabalho o objetivo de colocar a aplicação no ar foi atingido. Atualmente o projeto é \textit{open source}, aceitando contribuições externas, já que ainda há espaço para evoluir o sistema.  

O próximo passo do projeto é continuar o desenvolvimento dos requisitos gerados a partir da pesquisa. Por exemplo, um relógio marcando o tempo da reunião, avisos via e-mail e no próprio site sobre as próximas reuniões agendadas, sistema de sugestão de atividades baseado no histórico do time e usuários criarem novas atividades.

No entanto, o sistema já possui uma quantidade razoável de funcionalidades e consegue ser usado para reuniões de melhoria contínua. Assim, antes de se implementar os novos recursos, haverá uma divulgação do sistema para ser usado por mais times na comunidade.

 Os primeiros times a se levar em consideração na divulgação são os que responderam a pesquisa e mandaram o e-mail para contato. Esses times poderão refazer a pesquisa e averiguar se o \suricato{} ajudou no seu processo de melhoria contínua.

Além disso, a divulgação pode ser feita nas diversas listas da comunidade ágil. Isso verificará se o \suricato{} consegue ter uma boa aceitação na área. 

Para obter sugestões de melhorias para o sistema, outra estratégia é disponibilizar no site uma área para as pessoas mandarem \textit{feedbacks}.

Futuramente, está planejado realizar uma nova pesquisa que busca responder algumas perguntas que ainda estão em aberto, como:

\begin{enumerate}
	\item Os problemas de falta de engajamento em retrospectivas tem relação com a falta de variação das atividades?
	\item Como exatamente os times distribuídos fazem as suas reuniões de melhoria contínua? Que ferramentas eles utilizam e como elas poderiam ser melhoradas?
	\item Até que ponto a figura do facilitador pode ajudar com algumas das dificuldades encontradas em retrospectivas?
	\item Ortogonalmente ao assunto desse artigo, times distribuídos são realmente mais produtivos ou têm uma maior impressão de produtividade?
\end{enumerate}

Por fim, caso o Suricato se prove uma ferramenta útil na comunidade, será possível escrever alguns artigos para revistas e submeter palestras em eventos ágeis comentando sobre a ferramenta e os resultados obtidos. Também pretende-se submeter os resultados da pesquisa e pesquisas subsequentes para apreciação de conferências sobre agilidade que unem Academia e Indústria.