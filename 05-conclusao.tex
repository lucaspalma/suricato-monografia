\section{Conclusão}

Pesquisas indicam que um número crescente de times utiliza a prática de retrospectivas para promover melhoria contínua em seus ambientes. Nossa pesquisa corrobora com essas informações, o que nos deixa confiante a respeito do assunto escolhido como tema do estudo.

Nota-se que times ágeis apresentam uma grande consciência sobre problemas de comunicação, relacionamento e produtividade, embora a estrutura majoritária do time -- isto é, se o trabalho é feito localmente, distribuído no mesmo fuso horário, ou distribuído em diversos fusos horários -- influencie bastante nas questões enfrentadas pelos times pesquisados.

O intuito inicial do trabalho, desenvolver uma aplicação envolvendo diversas tecnologias e maximizar o aprendizado técnico durante o projeto, foi plenamente atingido. As expectativas de aprendizado no que diz respeito aos problemas enfrentados por times ágeis em seus processos de melhoria contínua, contudo, foram excedidas.

No início do projeto, a proposta do professor Dr. Alfredo Goldman era uma revisão sistemática sobre retrospectiva, o que proporcionou uma experiência diferente. Através dessa revisão, o conhecimento sobre métodos ágeis e em especial sobre melhoria contínua e retrospectiva tornou-se mais profundo.

A frustração de não encontrar artigos e pesquisas mais detalhados no assunto foi uma situação inesperada. O desenvolvimento de uma pesquisa foi algo novo e desafiador no trabalho. Nenhum dos autores tinha experiência em fazer uma pesquisa acadêmica, o que tornou o processo em um constante aprendizado. Se não fosse a ajuda e influência da Cecilia nas comunidades ágeis, teria sido muito mais difícil conseguir um conjunto relevantes de respostas.

Outra dificuldade encontrada foi analisar os dados coletados. Os membros não sabiam como interpretar e correlacionar as respostas. A equipe recebeu a ajuda de pessoas da Caelum, em especial do aluno de doutorado Mauricio Aniche, e conseguiu encontrar uma forma de analisar os dados, relacionar todas as respostas, e chegar a algumas conclusões. Estas foram vitais na hora de planejar os requisitos da aplicação.

Já havia uma ideia das tecnologias e ferramentas a se utilizar. Os autores já eram familiarizados com Java e Hibernate, então o desafio foi aprender a utilizar e resolver os problemas com as outras tecnologias incorporadas. O principal desafio foi em relação aos WebSockets e SpringSecurity, mas com a ajuda do senhor Alberto Tavares Souza foi possível ajustar toda a configuração do sistema.

Para o escopo do trabalho o objetivo de colocar a aplicação no ar foi atingido. Como o projeto é \textit{open source}, a equipe agora pode abrir para contribuições externas, já que apenas alguns requisitos foram implementados e ainda há espaço para evoluir.  

O próximo passo do projeto é continuar o desenvolvimento dos requisitos gerados a partir da pesquisa e planejados pela equipe. Por exemplo, um relógio marcando o tempo da reunião, avisos via e-mail e no próprio site sobre as próximas reuniões agendadas, sistema de sugestão de atividades baseado no histórico do time e  usuários criarem novas atividades.

No entanto, o sistema já possui uma quantidade razoável de funcionalidades e consegue ser usado para reuniões de melhoria contínua. Assim, os idealizadores do projeto acreditam que, neste momento, é mais interessante fazer uma pausa na implementação de novos recursos e começar a divulgar o sistema para ser usado por mais times além dos presentes na Caelum.

Primeiro, divulgaremos o sistema para os times que responderam a pesquisa e mandaram o e-mail para contato. Se esses times começarem a utilizar o Suricato é possível refazer a pesquisa e averiguar se o \suricato{} ajudou no seu processo de melhoria contínua.

Além disso, pretende-se também divulgar o sistema em diversas listas da comunidade ágil e verificar se o \suricato{} consegue ter uma boa aceitação. Também será disponibilizado no site uma área para as pessoas mandarem sugestões com melhorias que podem ser feitas no sistema ou novas funcionalidades.

Futuramente, ficou planejado realizar uma nova pesquisa que busca responder algumas perguntas que ficaram em aberto, como:

\begin{enumerate}
	\item Os problemas de falta de engajamento em retrospectivas tem relação com a falta de variação das atividades?
	\item Como exatamente os times distribuídos fazem as suas reuniões de melhoria contínua? Que ferramentas eles utilizam e como elas poderiam ser melhoradas?
	\item Até que ponto a figura do facilitador pode ajudar com algumas das dificuldades encontradas em retrospectivas?
	\item Ortogonalmente ao assunto desse artigo, times distribuídos são realmente mais produtivos ou têm uma maior impressão de produtividade?
\end{enumerate}

Por fim, caso o Suricato se prove uma ferramenta útil na comunidade, será possível escrever alguns artigos para revistas e submeter palestras em eventos ágeis comentando sobre a ferramenta e os resultados obtidos. Pretendemos, também, submeter os resultados da pesquisa e pesquisas subsequentes para apreciação de conferências sobre agilidade que unem Academia e Indústria.