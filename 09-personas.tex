\section{Personas}
\label{app:personas}

\begin{itemize}
	\item{Olivia (desenvolvedora)
	
	``Eu tenho que usar esse sistema... mas o que eu quero MESMO é que ele não atrapalhe meu
	trabalho''.
	
	Se fosse por ela, ela usava quadros na parede. Mas como sua equipe (inclusive cliente) não trabalha num mesmo ambiente físico, eles precisam usar o \calopsita.}
	\item{Nano (desenvolvedor remoto)
	
	``Quero saber o que já foi feito e qual é a próxima tarefa que eu tenho que fazer''.
	
	Nano mora no Havaí e trabalha com uma equipe no Brasil.}
	\item{Morelli (cliente)
	
	``Quero saber como anda o desenvolvimento da minha grande ideia''. 
	
	Morelli é um cliente que não sabe nada sobre desenvolvimento de software com uma grande ideia de um software para seu negócio.}
	\item{Fabs (colaborador distraído)
	
	``Eu quero bater o olho na página... e obter a informação que eu preciso''.
	
	Fabs algumas vezes é cliente, outras vezes é desenvolvedor. Sempre tem muitas coisas para fazer e não presta muita atenção em nada.}
	\item{Vinicius (\textit{coach} XP)
	
	``Quero ter uma visão geral do meu projeto XP''.}
	\item{Alexandre (\textit{scrum master})
	
	``Quero gerenciar meus projetos Scrum com facilidade''.}
	\item{Daniel (usuário leigo)
	
	``Muito legal esse negócio, mas como usa?''. 
	
	Daniel acaba de entrar num curso de computação e ouviu seus veteranos falando do \calopsita{}. Ele não sabe muita coisa de desenvolvimento, muito menos de métodos ágeis. Na verdade, ele mal sabe usar o computador mas é empolgado e quer aprender e ninguém pode ajudar.}
\end{itemize}
