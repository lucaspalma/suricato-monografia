\documentclass[titlepage,a4paper]{article} 
\usepackage{verbatim}
\usepackage[T1]{fontenc}
\usepackage[utf8]{inputenc}
\usepackage[brazil]{babel}
\usepackage{hyperref}
\usepackage{rotating}
\usepackage{listings}
\usepackage{color}
\usepackage{float}
\usepackage{amsmath, amsthm, amssymb}
\hypersetup{colorlinks=true,%
	citecolor=red,%
	linkcolor=red,%
	urlcolor=blue,%
	pdftex}
	
\newcommand{\opensource}{\textit{open source}}
\newcommand{\software}{\textit{software}}
\newcommand{\suricato}{Suricato}

\title{PARTE SUBJETIVA}
\author{Marcio Sakamoto Shibao}

\begin{document}

\maketitle
Entrei no IME por acaso. Sempre estive estive em dúvida do que cursar, comecei com Direito, depois prestei para Engenharia. Comecei a fazer o curso na Faculdade de Engenharia Industrial, porém sem muita certeza se era o que desejava. Acabei sendo chamado para o IME na terceira chamada e acabei me matriculando. Me matriculei no BCC e comecei o curso sem saber direito do que se tratava. Depois de muitos tropeços acabei me encontrando no curso e hoje, apesar das muitas dúvidas que tive, não me arrependo de ter escolhido o IME.

Tive a sorte de poder estagiar em lugares que ajudaram a crescer como pessoa. Um desses lugares, a Caelum. Felizmente, uma empresa que valoriza o aprendizado e que me ajudou a perceber a importância ou não de cada matéria do curso. Hoje consigo ver a formação que o IME me proporcionou e o crescimento que tive. Cresci muito como pessoa e profissional e cada dia dentro desse instituto contribuiu para isso.

O Suricato é a consolidação de tudo que aprendi durante esses anos e principalmente a afirmação de que fiz a escolha certa. Me sinto orgulhoso em ver o resultado que tivemos e tudo o que aprendemos. Nada disso seria possível sem os conhecimentos aprendidos na faculdade e nos estágios, sem esquecer de todas as pessoas que apoiaram e incentivaram para que esse tão esperado fim chegasse.
\section{Desafios e frustrações}

O maior desafio no curso foi arranjar motivação para continuar nele. O primeiro e segundo ano não foram exatamente como eu pensava. As matérias e os professores não foram muito empolgantes e devido a minha indecisão inicial, trocar de curso era um pensamento constante.

O fato de começar a estagiar a partir do terceiro semestre me deu um gás a mais, mas também fez com que o curso ficasse mais degastante e cansativo.

A grande frustração com certeza foi o segundo semestre. Nunca havia programado e depois de um primeiro semestre com uma introdução bem superficial, me senti completamente perdido com matérias mais desafiadores e professores menos interessados. Apesar disso acho que foi uma das maiores lições que aprendi no BCC, nem sempre teremos motivação e ânimo, mas temos que arranjar força para continuar e fazer o que é proposto.

Em relação ao Suricato, um desafio foi fazer o sistema suportar diversos usuários em tempo real. Além de toda dificuldade com o código, tivemos que aprender a configurar uma máquina, entender o servidor de aplicação e comprar e configurar um domínio para assim colocar nossa aplicação no ar.

Outro desafio foi em fazer telas com boa usabilidade. Nenhum de nós sabia como posicionar os elementos na tela e qual combinação de cor utilizar. A aparência inicial era ruim e mesmo depois de gastarmos tempo modificando, ainda não estava adequado. Somente com a ajuda de pessoas do trabalho é que conseguimos uma tela satisfatória.

Uma frustração com o Suricato foi a qualidade do código. No ambiente de trabalho sempre nos preocupamos em fazer testes e deixar o código organizado. Devido a falta de tempo, optamos por colocar primeiro a aplicação no ar e só depois pensarmos em refatorar.


\section{Disciplinas cursadas mais relevantes}

As pessoa que me acompanharam durante o curso, sabem a dificuldade que tive em entender a motivação da maioria das matérias. Somente no final do curso é possível perceber o uso de alguma delas. 

Para a realização do trabalho sinto que o estágio forneceu mais ferramentas que as matéria do IME, porém se não fosse algumas matérias teria muita dificuldade em adquirir conteúdo no estágio. Dentre elas gostaria de citar as que achei mais relevantes e motivantes durante todo o curso:

\begin{itemize}
	\item{\textbf{MAC0342 -- Laboratório de Programação Extrema}
	
    Para mim, uma das mais importantes de todo o curso pois era algo mais próximo do mercado de trabalho. Foi onde, pela primeira vez, me deparei com um projeto com um cliente real. Foi uma matéria que proporcionou um grande desafio: o de trabalhar em grupo com pessoas que não sabiam trabalhar em grupo. Ter sido o coach da minha equipe me permitiu crescer muito como pessoa.}
	\item{\textbf{MAC0338 -- Análise de Algoritmos}
	
	Para mim uma das matérias mais difíceis e desafiadoras do curso, mas também uma das melhores. Foi a primeira matéria a qual me dediquei e esforcei com prazer.}
	\item{\textbf{MAC0448 -- Programação para Redes de Computador}
	
	A única matéria que tive contato com programação para redes de computador. Também foi um grande desafio, por ser um matéria com grande conteúdo e EPs realmente desafiadores.}
	\item{\textbf{MAC0439 -- Laboratório de Banco de Dados}
	
    Uma das matérias em que mais aprendi. Muito mais completa e estruturada que a matéria requisito. Foi possível rever e aprender conceitos deixados de lado, além de colocar em prática todo o conceito aprendido.}
\end{itemize}

\section{Futuro}

Pretendo continuar o trabalho na Caelum e dar sequência ao Suricato. Ainda não sei quais áreas pretendo seguir e me apronfundar, mas pretendo estudar e focar na parte de metodologias e métodos ágeis.

Dentro da Caelum pretendo explorar programação mobile e aprofundar meus conhecimentos em front-end. Acho importante poder ter contato com diversas linguagens diferente de programação para depois poder escolher uma para me aprofundar.

\section{Agradecimentos}

Muitas pessoas merecem agradecimentos e não seria possível citar todas aqui. Então cito pontualmente aquelas que fizeram grande diferença nesses anos de IME.

A todos os meus amigos que me apoiaram e incentivaram durante todo o curso. Em especial a Mayra que foi quem mais cobrou e torceu durante os longos anos dentro do IME e me ajudou nos momentos mais difíceis dentro e fora da faculdade.

Ao Lucas, minha dupla de projeto, que me ajudou não somente no desenvolvimento do projeto, mas em diversas matérias ao longo do curso. A Ceci, co-orientadora do projeto, que defendeu e apoio nossa ideia esteve sempre presente em todo o desenvolvimento. Ao Prof. Dr. Alfredo Goldman, por aceitar nossa ideia, nos incentivar a melhorar nossos conhecimentos e, mesmo à distância nos dar sugestões e criticas.

A todas as pessoas da Caelum, pelo patrocínio do projeto, pelos conselhos e pela amizade. Em especial aos irmãos Paulo e Guilherme Silveira, pela oportunidade de trabalhar na empresa. 

Agradeço aos funcionários da seção de informática e da seção de alunos do IME que sempre fizeram o possível para ajudar.


\end{document}

