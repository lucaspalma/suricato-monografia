\documentclass[titlepage,a4paper]{article} 
\usepackage{verbatim}
\usepackage[T1]{fontenc}
\usepackage[utf8]{inputenc}
\usepackage[brazil]{babel}
\usepackage{hyperref}
\usepackage{rotating}
\usepackage{listings}
\usepackage{color}
\usepackage{float}
\usepackage{amsmath, amsthm, amssymb}
\hypersetup{colorlinks=true,%
	citecolor=red,%
	linkcolor=red,%
	urlcolor=blue,%
	pdftex}
	
\newcommand{\opensource}{\textit{open source}}
\newcommand{\software}{\textit{software}}
\newcommand{\suricato}{Suricato}

\title{PARTE SUBJETIVA}
\author{Lucas Takeshi Rodrigues Palma}

\begin{document}

\maketitle

Desde o inicio da minha formação educacional, nunca tinha enfrentado grandes dificuldades quando o assunto se tratava de ensino. Até o colegial, eu sabia que com um mínimo de esforço era possível ir bem nas matérias. Talvez o primeiro grande obstáculo neste sentido tenha sido a FUVEST. No entanto, como cursei um colegial voltado para vestibulares e o curso de ciência da computação não é um dos mais difíceis de entrar, passar na USP acabou sendo relativamente tranquilo.

Então meu primeiro grande desafio acadêmico foi o IME, tanto pelos motivos corretos quanto pelos errados, que serão apontados mais a frente. Por isso, conseguir me formar é um motivo de grande alegria e alívio.

Olhando para trás, eu vejo o quanto evolui nestes últimos 5 anos de BCC. Para uma pessoa que entrou na faculdade sem saber escrever uma linha de código para um profissional que, hoje em dia, ensina pessoas a entrarem no mundo da programação é um crescimento que eu não imaginava no início do curso. A base que tornou tudo isso possível acabou sendo o IME.

Termino esta fase da minha vida criando o \suricato{}, um projeto do qual me orgulho muito e vejo com grande potencial para ajudar toda a comunidade ágil. Ele é o resultado de tudo que aprendi ao longo desses anos e não vou pará-lo por aqui.

Este projeto não seria possível sem o estágio que fiz durante a minha graduação. Por mais que muitos professores do IME, erronea e ingenuamente, defendam que os alunos não devem estagiar durante a graduação, não tenho dúvidas que esta foi a decisão mais certa que eu fiz em toda a minha vida. Por mais que o estágio tenha atrasado em um ano o tempo que levei para me formar em relação ao tempo "ideal", o aprendizado que obtive foi gigantesco.

Não tenho palavras para agradecer a Caelum que me ofereceu essa oportunidade e, sem dúvidas, me salvou como um profissional da programação. Lembro que comecei o estágio na metade do terceiro ano, uma época em que não podia estar mais desmotivado com relação a computação e a USP. Depois de 3 anos e meio com aulas puramente teóricas e professores, em sua grande maioria, completamente desmotivados em dar aula para a graduação, eu começava questionar se tinha escolhido a profissão correta e pensava em desistir. 

Hoje percebo como tive a sorte de receber o e-mail do Paulo Silveira me oferecendo uma oportunidade de estágio. Como a empresa dá uma importância muito grande para o ensino, pude aprender como a programação é na prática. Graças a Caelum que hoje tenho o conhecimento sobre como é o mercado, quais as tecnologias que as empresas utilizam e o que é programação. Mesmo assim, nenhum desses pontos foi mais significativo do que o crescimento pessoal. Saber lidar com pessoas e trabalhar em equipe são experiências que não teria sem o estágio.

Por isso, percebo que minha graduação seria incompleta sem a Caelum. Não adianta somente a teoria que o IME dá para seus alunos. Junto dela deve vir a prática, o que não se resume a fazer EPs como muitos na instituição acreditam. Os EPs são importantes, já que virar noites em claro para resolvê-los ajudam a formar o caráter e a aprender a se virar. No entanto, eles não refletem os desafios que são enfrentados de verdade no dia a dia do trabalho.

\section{Desafios e frustrações}

Foi grande a quantidade de desafios e frustrações que tive durante o curso. Muitos deles ajudaram na minha formação, já que graças a isso hoje eu sei aprender os mais diversos assuntos sozinho. No entanto, isso não significa que isso possa ser usado como justificativa para algumas situações que acontecem durante a graduação.

\subsubsection*{Desafios}

Dois desafios com relação aos professores eu quero deixar bem registrados, pois acho ambos inadmissíveis acontecerem em uma instituição como a USP. O primeiro foi grande quantidade de professores que chegam atrasados para as aulas. Dado que faz parte do \textbf{trabalho} de um doutor da USP dar aulas, deveria  ser obrigatório ele estar na sala no horário estabelecido, independente se a pessoa gosta ou não de lecionar.

O segundo ponto é que muitos professores precisam aprender a escrever na lousa. Foi grande o número de professores com quem tive aula que a letra era inelegível. Ter vontade de dar um caderno de caligrafia para mais de um professor da USP era um sentimento que eu não imaginava ter no começo da graduação. Além disso, muitos também escrevem com um tamanho de letra muito pequeno, principalmente nas maiores salas do IME. Para alguém como eu que possui tanto miopia quanto astigmatismo, mesmo usando óculos senti dificuldade durante toda a graduação de entender o que escreviam no quadro.

Outro grande desafio foi organizar os horários da faculdade com os meus outros compromissos, principalmente no quesito social. A grande quantidade de matérias que tive nos dois primeiros anos acabava exigindo quase que uma dedicação exclusiva ao curso. O desafio foi ainda maior depois que eu comecei o estágio no terceiro ano, mesmo com o número de matérias reduzido, mas ainda conseguia superá-lo.

Contudo, nesse último semestre acabei sendo superado e acabei me abdicando um pouco da vida social. Com matérias para concluir no IME, a efetivação e, consequentemente, as maiores responsabilidades no trabalho e desenvolvimento do \suricato{}, não havia muito tempo disponível para outras atividades pessoais.

\subsubsection*{Frustrações}

Com relação ao \suricato{}, tenho apenas duas frustrações. Uma é que o layout do site ainda não está o ideal. Ele melhorou muito nas últimas semanas, mas ainda não está visualmente agradável para mim. Essa frustração não é tão grande pois algumas pessoas já se prontificaram para nos ajudar a melhorar o layout depois que o TCC terminar.

Minha segunda frustração com o projeto é que o \suricato{} ainda não foi utilizado por outros times além dos na Caelum. Estamos segurando a divulgação do projeto na comunidade por conta de melhorias que ainda precisam ser feitas no design do sistema e também devido a algumas instabilidades que acontecem no servidor de produção, que eventualmente cai. Espero resolver esses problemas o quanto antes para começar a apresentar o sistema para outras pessoas.

Já com relação ao curso foram algumas as minhas frustrações.  A primeira delas aconteceu em Física I com o dr. Hideaki Miyake. Estava no segundo semestre do primeiro ano e fiquei consternado em como uma aula da USP poderia ser tão ruim. Como fiz o colegial em uma escola muito boa, ver um professor de física não conseguir demonstrar uma fórmula física em quase todas as suas aulas era espantoso. Claramente não havia uma preocupação com o preparo da aula e, consequentemente, com a qualidade do que estava sendo ensinado.

Mais para frente percebi que a falta de preocupação com as qualidade das aulas não era uma atitude exclusiva do professor Miyake. Alguns professores do IME agem da mesma forma e faço questão de citar alguns deles:

\begin{itemize}
	\item Sistemas operacionas com o dr. Alan Mitchell Durham: eu não sentiria orgulho em dar uma aula como a que tive nesta matéria. A incapacidade de manter um tom de voz audível, a interrupção constante da explicação para corrigir algum erro de português nos slides e a péssima didática para explicar o assunto tornaram essa a pior aula que já tive na minha vida.
	\item Engenharia de \textit{Software} com a dra. Ana Cristina Vieira de Melo: como sempre estive envolvido mais na área de desenvolvimento de sistemas, fiquei decepcionado em como esse curso está desatualizado. No decorrer da monografia fica explícito que trabalho com métodos ágeis e é abismal o crescimento que a agilidade teve no mercado nos últimos anos. Ver o IME parado no tempo, explicando apenas modelos antiquados, como o waterfall, é motivo de grande decepção.
	\item Princípios de Interação Humano Computador com dr. Carlos Hitoshi Morimoto: essa é a matéria que define o conceito de frustração. Como atualmente sou instrutor na Caelum, eu sei que um dos fundamentos didáticos é que nada desmotiva mais um aluno na aula do que o não. Utilizar de palavras negativas a para contradizer a resposta de um aluno fazem com o ele se sinta inseguro para falar algo novamente e, em alguns casos, envergonhado por ter seu erro apontado para toda a turma. Na aula do dr. Hitoshi eu cheguei a perder a conta de quantas vezes ele fala não para seus alunos, mesmo que a resposta estivesse correta ou o aluno estivesse falando a mesma coisa que ele. Era visível a frustração dos alunos durante as aulas e, principalmente, durante o projeto. Na matéria, uma profissional da área de psicologia e pedagogia foi convidada para apresentar uma proposta de trabalho para a disciplina. A ideia era desenvolver uma aplicação para crianças especiais com paralisia nos músculos do corpo. No entanto, ao logo da matéria ficou claro que a ideia do projeto não era satisfazer a cliente, mas sim o dr Hitoshi. Chegou ao ponto de ser realizado um \textit{hackathon} onde a cliente foi convidada e, durante uma discussão sobre o projeto, eu tive que ouvir como argumento do dr. que as ideias e opiniões da cliente não eram para serem levadas em consideração, mas sim o que ele estava falando que deveria ser feito. Um detalhe importante é que a cliente se encontrava ao meu lado e ela se sentiu incomodada. Para alguem que segue os princípios da agilidade, em que uma das coisas mais importantes é trazer o cliente para junto do time de desenvolvimento e satisfazê-lo, ouvir tais palavras provocaram a maior frustração e desgosto que tive durante toda a graduação. Falar que a opiniões de uma pessoa não são importantes para um projeto já saem do ambito de metodologias da computação e já entram na área de respeitar as outras pessoas. E um professor da USP agir desta forma é incabível.
	\item POO?????????????????????????????????????????????//
\end{itemize}

Estes foram apenas as principais frustrações. No entanto, acho que tudo se resume a falta de didática e a desatualização que a grande maioria dos professores da computação pecam. 

Fico abismado em como a grnade maioria dos professores não tem a menor preocupação em melhorar. Talvez devido ao cargo que obtiveram e a aparente inexistência de uma cobrança pelo que acontece em suas aulas, esse cenário nunca venha a mudar. No entanto, agora no fim da graduação não posso deixar explícito como é frustrante ver essas coisas, principalemnte no quesito atualização.

Melhorar a didática das pessoas é um processo longo e difícil. A menos que haja um mínimo de iniciativa da pessoa e vergonha na cara que ela está dando uma aula ruim e precisa melhorar pelo bem dos alunos, a pessoa realmente não consegue melhorar sua didática.

No entanto, todos os  doutores da USP deveriam ter o mínimo de noção que o que era usado a 20, 30 ou 40 anos já não é mais a realidade porque as coisas evoluem. Principalmente quando estamos falando de computação, em que essa evolução se da ainda mais rápido do que em outras áreas. Ver tantas pessoas paradas no tempo, muito porque nunca sequer tiveram contato com o mercado de verdade, chega a ser vergonhoso.

\section{Disciplinas cursadas mais relevantes}

Mesmo sentindo frustração em diversas matérias, algumas disciplinas foram realmente importantes para minha formação. As que gostaria de destacar que foram realmente marcantes foram:

\begin{itemize}
	\item{\textbf{MAC0323 -- Estruturas de Dados -- Dr. Yoshiharu Kohayakawa} 

		Essa foi uma das matérias que me transformou como programador, pois foi nela em que minha lógica de programação realmente evoluiu. É interessante o quanto esta matéria faz parte do meu dia a dia, tanto no meu código quanto nas minhas aulas quando preciso explicar estruturas como listas ligadas para os alunos. Além disso, a preocupação do professor Yoshiharu em preparar as aulas e a 
		ótimo fluxo que o seu raciocínio seguia durante as aulas era algo que se destacava em relação as demais disciplinas que eu cursava na época.}
	\item{\textbf{MAC0342 -- Laboratório de Programação Extrema -- Dr. Alfredo Goldman}

		Única matéria que fiz em toda a minha graduação que refletia um pouco o que aocntece no mercado,  devido a preocupação do professor Gold em encontrar projetos de diversas empresas para serem usados na dosciplina. Na época eu já estava fazendo estágio na Caelum, então pessoalmente a disciplina serviu mais como um complemento para meu aprendizado sobre agilidade e desenvolvimento em equipe. Mesmo assim, acho vital a existência de uma matéria neste formato, já que nem todos os alunos do IME terão contato com o mercado dado a cultura implantada pelos professores.}
	\item{\textbf{Redes -- Dr. Daniel Macêdo Batista}

		Esta matéria me chamou a atenção muito por conta das aulas do professor Daniel. É incrível como uma aula preparada com cuidado e uma didática adequada fazem diferença no aprendizado. Mesmo não sendo a área com a qual eu pretenda trabalhar, Redes foi provavelmente uma das matérias em que eu mais aprendi, muito por conta da qualidade da aula. É bom ver que os professores mais jovens do IME estão tendo essa preocupação, o que me deixa com uma certa esperança com o futuro do instituto}
\end{itemize}

\section{Futuro}

No próximo ano, pretendo seguir com o trabalho na Caelum, provavelmente focando na parte de métodos ágeis e engenharia de \software{} contemporânea -- métodos ágeis e tradicionais e o porquê da migração e afins. Comecei esse trabalho já esse ano, mas ainda há muito o que aprender.

Pretendo também estudar a área de Interação Homem-Computador pela literatura disponível na área e colocar os conceitos em prática nos projetos em que participar. Sinto que essa é uma área em grande \textit{deficit} no mercado nacional e ela desperta meu interesse.

Uma vez estudadas ambas as áreas, tenho a intenção de investir em um curso de especialização profissional na que mais me agradar -- provavelmente cursos no exterior, talvez algum de ensino à distância.

\section{Agradecimentos}

Muitas pessoas merecem agradecimentos e não seria possível colocá-las todas aqui. Então, pontualmente, falo de algumas pessoas que fizeram uma grande diferença nesses anos de IME. 

Começo pela Cris Sato porque é culpa dela que, entre USP e Unicamp, decidi ficar no IME. O tempo que ela investiu em, durante meu colegial, me trazer para o IME, apresentar pessoas, o ambiente, a importantíssima máquina de café, me fizeram desistir de um plano já antigo de ir para Campinas.

Aos veteranos também de 2003, Mari e Hugo, pelas inúmeras e valiosas conversas sobre o curso, a vida e tudo o mais. Pelos convites para escaladas (importante se manter saudável) e pela amizade.

Ao professor Alfredo Goldman, que comprou a nossa idéia, nos orientou, foi às apresentações do TCC assistir à nossa e participou de uma reunião em plena sexta-feira à noite em um café. Muito obrigado.

Aos amigos mais antigos, nos quais incluo meus pais, que aguentaram muito tempo de sumiço, nervosismos e desesperos. Agradecimentos e desculpas a esses. Em especial, um agradecimento ao Lucas Frenay, que sempre me serviu de inspiração e exemplo.

À Dilma, supracitada, pela oportunidade de conhecer Nova York e o centro de pesquisas da IBM, pelas dicas de musicais, pelo carinho e pelo exemplo de pessoa.

Também devem ser citados Andrea e Jimi Xenidis, que infelizmente não poderão ler esse texto em português, mas que aparecem aqui por terem dedicado tantos finais de semana a nos fazer sentir em casa, enquanto nos Estados Unidos.

À Caelum, pelo aprendizado, sim. Mas sobretudo pelo espírito de equipe e por ter sido tão bem recebida. Voltando da IBM, saudosa da Caelum, fui recebida com ``bem vinda de volta à casa paterna''. De fato, é o que essa empresa é para mim. Os amigos que conheci na Caelum e com quem aprendi muito também têm seu espaço aqui.

Aos companheiros de projeto, cúmplices de finais de semana e guerreiros em discussões sobre pontos controversos no projeto e na tecnologia. Crescemos muito juntos.

E principalmente aos professores presentes, que mostraram que se importam com o curso e querem continuar melhorando-o sempre. É pelas iniciativa desses que mantemos o orgulho de estudar ou ter estudado nessa universidade. Citando o professor Marco de Canto Coral/ECA:

\begin{quote}
	\textit{``Eu cuido do meu diploma todo dia. E é função de cada um de vocês fazer isso também.''}
\end{quote}

Concordo, é nossa função. E espero fazer jus a ele em cada curso que eu lecionar na Caelum, projeto \opensource{} no qual participar ou texto que publicar. Dessa vez, comemoro minha formatura.
\end{document}
