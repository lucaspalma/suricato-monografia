\documentclass[titlepage,a4paper]{article} 
\usepackage{verbatim}
\usepackage[T1]{fontenc}
\usepackage[utf8]{inputenc}
\usepackage[brazil]{babel}
\usepackage{hyperref}
\usepackage{rotating}
\usepackage{listings}
\usepackage{color}
\usepackage{float}
\usepackage{amsmath, amsthm, amssymb}
\hypersetup{colorlinks=true,%
	citecolor=red,%
	linkcolor=red,%
	urlcolor=blue,%
	pdftex}
	
\newcommand{\opensource}{\textit{open source}}
\newcommand{\software}{\textit{software}}
\newcommand{\suricato}{Suricato}

\title{PARTE SUBJETIVA}
\author{Lucas Takeshi Rodrigues Palma}

\begin{document}

\maketitle

Desde o inicio da minha formação educacional, nunca tinha enfrentado grandes dificuldades quando o assunto se tratava de ensino. Até o colegial, eu sabia que com um mínimo de esforço era possível ir bem nas matérias. Talvez o primeiro grande obstáculo neste sentido tenha sido a FUVEST. No entanto, como cursei um colegial voltado para vestibulares e o curso de ciência da computação não é um dos mais difíceis de passar, entrar na USP acabou sendo relativamente tranquilo.

Então meu primeiro grande desafio acadêmico foi o IME, tanto pelos motivos corretos quanto pelos errados, que serão apontados mais a frente. Por isso, conseguir me formar é um motivo de grande alegria e alívio.

Olhando para trás, eu vejo o quanto evolui nestes últimos 5 anos de BCC. Para uma pessoa que entrou na faculdade sem saber escrever uma linha de código para um profissional que, hoje em dia, ensina programação às pessoas um crescimento que eu não imaginava no início do curso. A base que tornou tudo isso possível foi o IME.

Termino esta fase da minha vida criando o \suricato{}, um projeto do qual me orgulho muito e vejo com grande potencial para ajudar toda a comunidade ágil. Ele é o resultado de tudo que aprendi ao longo desses anos e não vou pará-lo por aqui.

Este projeto não seria possível sem o estágio que fiz durante a minha graduação. Por mais que muitos professores do IME, errônea e ingenuamente, defendam que os alunos não devem estagiar durante a graduação, não tenho dúvidas que esta foi a decisão mais certa que eu fiz em toda a minha vida. Por mais que o estágio tenha atrasado em um ano o tempo que levei para me formar em relação ao tempo "ideal", o aprendizado que obtive foi gigantesco.

Não tenho palavras para agradecer a Caelum que me ofereceu essa oportunidade e, sem dúvidas, me salvou como um profissional da programação. Lembro que comecei o estágio na metade do terceiro ano, uma época em que não podia estar mais desmotivado com relação a computação e a USP. Depois de 3 anos e meio com aulas puramente teóricas e muitos professores completamente desmotivados em dar aula para a graduação, eu começava questionar se tinha escolhido a profissão correta e pensava em desistir. 

Hoje percebo como tive a sorte de receber o e-mail do Paulo Silveira me oferecendo uma oportunidade de estágio. Como a empresa dá uma importância muito grande para o ensino, pude aprender como é o mercado de trabalho, quais tecnologias que as empresas utilizam e o que é programação na prática. Mesmo assim, nenhum desses pontos foi mais significativo do que o crescimento pessoal. Saber lidar com pessoas e trabalhar em equipe são experiências que não teria sem o estágio.

Por isso, noto que minha graduação seria incompleta sem a Caelum. Não adianta somente a teoria que o IME dá para seus alunos. Junto dela deve vir a prática, o que não se resume a fazer EPs como muitos na instituição acreditam. Os EPs são importantes, já que virar noites em claro para resolvê-los ajudam a formar o caráter e a aprender a se virar. No entanto, eles não refletem os desafios que são enfrentados de verdade no dia a dia do trabalho.

\section{Desafios e frustrações}

Foi grande a quantidade de desafios e frustrações que tive durante o curso. Muitos deles ajudaram na minha formação, já que graças a isso hoje eu sei aprender os mais diversos assuntos sozinho. No entanto, não significa que isso possa ser usado como justificativa para algumas situações que acontecem na graduação.

\subsubsection*{Desafios}

Existem dois desafios com relação aos professores que eu quero deixar bem registrados, pois acho ambos inadmissíveis acontecerem em uma instituição como a USP. O primeiro foi a grande quantidade de professores que chegam atrasados para as aulas. Dado que faz parte do \textbf{trabalho} de um doutor da USP dar aulas, deveria ser obrigatório ele estar na sala no horário estabelecido, independente se a pessoa gosta ou não de lecionar.

O segundo ponto é que muitos professores precisam aprender a escrever na lousa. Foi grande o número de professores com quem tive aula que a letra era inelegível. Ter vontade de dar um caderno de caligrafia para mais de um professor da USP era um sentimento que eu não imaginava ter no começo da graduação. Além disso, muitos também escrevem com um tamanho de letra muito pequeno, principalmente nas maiores salas do IME. Para alguém como eu que possui tanto miopia quanto astigmatismo, mesmo usando óculos senti dificuldade durante toda a graduação de entender o que escreviam no quadro.

Outro grande desafio foi organizar os horários da faculdade com os meus outros compromissos, principalmente no quesito social. A grande quantidade de matérias que tive nos dois primeiros anos acabava exigindo quase que uma dedicação exclusiva ao curso. O desafio foi ainda maior depois que eu comecei o estágio no terceiro ano, mas ainda conseguia superá-lo.

Contudo, nesse último semestre fui superado e acabei me abdicando um pouco da vida social. Com matérias para concluir no IME, a efetivação e, consequentemente, as maiores responsabilidades no trabalho e o desenvolvimento do \suricato{}, não havia muito tempo disponível para outras atividades pessoais.

\subsubsection*{Frustrações}

Com relação ao \suricato{}, tenho apenas duas frustrações. Uma é que o layout do site ainda não está o ideal. Ele melhorou muito nas últimas semanas, mas ainda não está visualmente agradável para mim. Essa frustração não é tão grande, pois algumas pessoas já se prontificaram a nos ajudar a melhorar o layout depois que o TCC terminar.

Minha segunda frustração com o projeto é que o \suricato{} ainda não foi utilizado por outros times além dos que estão na Caelum. Estamos segurando a divulgação do projeto na comunidade por conta de melhorias que ainda precisam ser feitas no design e a algumas instabilidades que acontecem no servidor de produção, que eventualmente cai. Espero resolver esses problemas o quanto antes para começar a apresentar o sistema para outras pessoas.

Já com relação ao curso foram algumas as minhas frustrações.  A primeira delas aconteceu em Física I com o dr. Hideaki Miyake. Estava no segundo semestre do primeiro ano e fiquei consternado em como uma aula da USP poderia ser tão ruim. Como fiz o colegial em uma escola muito boa, ver um professor de física não conseguir demonstrar uma fórmula física em quase todas as suas aulas era espantoso. Claramente não havia uma preocupação com o preparo da aula e, consequentemente, com a qualidade do que estava sendo ensinado.

Mais para frente percebi que a falta de preocupação com as qualidade das aulas não era uma atitude exclusiva do professor Miyake. Alguns professores do IME agem da mesma forma e faço questão de citar alguns deles:

\begin{itemize}
	\item{\textbf{MAC0422 -- Sistemas Operacionais -- Dr. Alan Mitchell Durham}

	Eu não sentiria orgulho em dar uma aula como a que tive nesta matéria. A incapacidade de manter um tom de voz audível, a interrupção constante da explicação para corrigir algum erro de português nos slides e a péssima didática para explicar o assunto tornaram essa a pior aula que já tive na minha vida.}

	\item{\textbf{MAC0332 -- Engenharia de \textit{Software} -- Dra. Ana Cristina Vieira de Melo}

	Como sempre estive envolvido mais na área de desenvolvimento de sistemas, fiquei decepcionado em como a disciplina está desatualizada. No decorrer da monografia fica explícito que trabalho com métodos ágeis e é abismal o crescimento que a agilidade teve no mercado nos últimos anos. Ver o IME parado no tempo, explicando apenas modelos antiquados, como o \textit{waterfall}, é motivo de grande decepção.}

	\item{\textbf{MAC0446 -- Princípios de Interação Humano-computador -- Dr. Carlos Hitoshi Morimoto}

	Essa matéria definiu o conceito de frustração. Como sou instrutor na Caelum, eu sei que um dos fundamentos didáticos é que nada desmotiva mais um aluno na aula do que o não. Usar uma palavra negativa para contradizer a resposta de um aluno faz ele se sentir inseguro para falar algo novamente e, em alguns casos, envergonhado por ter seu erro apontado para toda a turma. Nesta aula do dr. Hitoshi eu cheguei a perder a conta de quantas vezes ele falou não para seus alunos, mesmo que a resposta estivesse correta ou o aluno falasse a mesma coisa que ele. Era visível a frustração dos alunos durante as aulas e, principalmente, durante o projeto da matéria. 
	\newline
	Na metade da disciplina uma profissional que trabalha com crianças especiais foi convidada para apresentar uma proposta para o trabalho. A ideia era desenvolver uma aplicação para crianças com amiotrofia espinhal. No entanto, no decorrer da matéria ficou claro que a ideia do projeto não era satisfazer a cliente, mas sim o dr Hitoshi. Durante um \textit{hackathon} que ocorreu no fim do semestre e acabei tendo uma discussão com o dr. Hitoshi sobre o projeto. Como já estava me envolvendo com agilidade, usei de alguns conceitos da metodologia para envolver o cliente no projeto. Como ela era a pessoa que tinha um contato com essas crianças, minha equipe manteve contato constante com a profissional sobre o projeto. O constante \textit{feedback} que ela nos dava era vital para saber se a aplicação seria realmente útil.
	\newline
	Durante o \textit{hackathon} o dr. Hitoshi veio questionar o por quê da aplicação estar daquela forma e não ser do jeito que ele esperava. Quando argumentei que havia sido um trabalho em conjunto da equipe recebendo informações, ideias e aprovações da cliente, acabei ouvindo como argumento do doutor que as ideias e opiniões da profissional não deveriam ser levadas em consideração. Eu deveria desenvolver apenas o que ele ele tinha em mente que seria uma boa aplicação para essa crianças com as quais ele nunca teve contato. Para piorar, a cliente se encontrava ao meu lado incomodada com a situação. Como eu acredito nos princípios da agilidade, em que uma das coisas mais importantes é trazer o cliente para junto dos desenvolvedores e satisfazê-lo, ouvir tais palavras provocaram a maior frustração e desgosto de toda a graduação. Um professor da USP faltar com tamanho respeito com outra pessoa é incabível.}
\end{itemize}

\section{Disciplinas cursadas mais relevantes}

Mesmo sentindo frustração em diversas matérias, algumas foram realmente importantes para minha formação. Gostaria de destacar aquelas que foram realmente marcantes:

\begin{itemize}
	\item{\textbf{MAC0323 -- Estruturas de Dados -- Dr. Yoshiharu Kohayakawa} 

		Essa foi uma das matérias que me transformou como programador, pois foi nela em que minha lógica de programação realmente evoluiu. É interessante o quanto esta disciplina faz parte do meu dia a dia, presente tanto no código que escrevo todos os dias quanto nas minhas aulas quando preciso explicar estruturas como listas ligadas para os alunos. Além disso, a preocupação do professor Yoshiharu em preparar as aulas e o ótimo fluxo que o seu raciocínio seguia durante as aulas era algo que se destacava.}
	\item{\textbf{MAC0342 -- Laboratório de Programação Extrema -- Dr. Alfredo Goldman}

		Única matéria que fiz em toda a minha graduação que refletia um pouco o que acontece no mercado,  devido a preocupação do professor Alfredo em encontrar projetos de diversas empresas para serem usados na disciplina. Na época eu já estava estagiando na Caelum, então a disciplina me serviu mais como um complemento para meu aprendizado sobre agilidade e desenvolvimento em equipe. Mesmo assim, acho vital a existência de uma matéria neste formato, já que não são todos os alunos do IME terão contato com o mercado dado a cultura implantada pelos professores.}
	\item{\textbf{MAC0448 -- Programação para Redes de Computadores -- Dr. Daniel Macêdo Batista}

		Esta disciplina me chamou a atenção por conta das aulas do professor Daniel. É incrível como uma aula preparada com cuidado e uma didática adequada fazem a diferença no aprendizado. Mesmo não sendo uma área que pretendo trabalhar, redes foi uma das matérias em que eu mais aprendi devido a qualidade da aula. É bom ver que os professores mais novos do IME tem essa preocupação. Me deixa com uma certa esperança com o futuro das aulas do instituto.}
\end{itemize}

\section{Futuro}

Pelos próximos anos pretendo continuar na Caelum. Vou me focar cada vez mais em agilidade e também quero cada vez melhorar a minha didática e dar mais aulas. Ser instrutor me permite mudar a vida das pessoas através do ensino. Essa é uma experiência extremamente gratificante e motivo de muito orgulho.

Além disso, pretendo estudar os conceitos de \textit{User Experience}. É uma área que me desperta interesse, já que ela pode ser aplicada em conjunto com os métodos ágeis. Existem alguns estudos na comunidade de pessoas tentando juntar ambas as áreas, mas vejo que o processo ainda não está completo.

Futuramente, tenho planos de voltar para a área acadêmica e fazer um mestrado. No entanto, sinto que preciso de alguns anos longe da academia depois de 5 anos de graduação no IME.

\section{Agradecimentos}

Existem muitas pessoas que eu gostaria de agradecer e que tornaram possível a conclusão da minha graduação e, direta ou indiretamente, a criação do \suricato{}.

A primeira pessoa que merece lugar nesta lista é meu amigo Rafael Garcia. Se estou concluindo minha graduação na USP é por conta dele. Ao longo dos últimos 5 anos foram incontáveis as vezes que recebi seu apoio e incentivo quando pensava em desistir ou estava a beira da depressão frente aos momentos de tristeza e estresse. Por isso fica aqui o meu mais profundo obrigado.

Do IME, existem dois professores que gostaria muito de deixar meu mais sincero agradecimento. Primeiro, à professora Cristina Gomes Fernandes, que me indicou para estagiar na Caelum depois que eu assisti sua aula de análise de algoritmos. Segundo, ao professor Alfredo Goldman, que aceitou nos orientar e deu uma grande auxílio no momento mais crítico do trabalho que foi a decisão de realizar uma pesquisa na comunidade. Muito obrigado aos dois.

Também fica aqui o meu agradecimento e pedido de desculpas aos meus pais, familiares e amigos dos tempos de colégio, que tiveram que me aguentar nesse ano de ausência e, muitas vezes, surtos. 

Não poderia faltar o meu mais profundo agradecimento à Caelum. Chega a ser difícil até mesmo expressar o quanto esta empresa significa para mim. Se sou o profissional que sou hoje é tudo graças a Caelum e seus funcionários, com quem continuo a aprender dia após dia. 

Dentre as pessoas que trabalham comigo, algumas merecem uma menção pessoal. Começo pela Cecilia Fernandes, que aceitou coorientar este trabalho. Sem as constantes chamadas de atenção e a procura constante pela excelência do trabalho, o \suricato{} não chegaria no nível que ele está hoje. Além disso, é uma pessoa exemplar, que particularmente me serve como espelho de profissional. Sinto orgulho de fazer parte do mesmo time da Ceci na Caelum e poder aprender constantemente com ela.

Outras duas pessoas que gostaria de agradecer são os instrutores Alberto Tavares e Joviane Jardim, que acreditaram no meu potencial e me treinaram para ser um instrutor na Caelum. O conhecimento que adquiri durante este processo é o que tornou possível utilizar tantas tecnologias diferentes no \suricato{}.

Não poderia faltar o meu obrigado ao Marcio Shibao. Mesmo depois de tantas dificuldades, discussões infinitas e estresse ao longo dos últimos meses, nós finalmente conseguimos concluir nosso TCC.

Por fim, fica meu agradecimento ao IME e seus professores em geral. Mesmo em meio a tantos conflitos e contradições durante a graduação, considero que tive uma ótima formação. Com certeza o conhecimento que obtive ao longo desses 5 anos me servirá como uma ótima base para o que está por vir. Espero utilizar este conhecimento em cada curso na Caelum e mudar a vida de mais pessoas com ele. E quem sabe em um futuro eu retorne para contribuir mais com a instituição.

No momento, é com grande alívio e alegria que concluo a minha graduação no BCC.
\end{document}
