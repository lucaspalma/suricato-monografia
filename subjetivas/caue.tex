\documentclass[titlepage,a4paper]{article} 
\usepackage{verbatim}
\usepackage[T1]{fontenc}
\usepackage[utf8]{inputenc}
\usepackage[brazil]{babel}
\usepackage{hyperref}
\usepackage{rotating}
\usepackage{listings}
\usepackage{color}
\usepackage{float}
\usepackage{amsmath, amsthm, amssymb}
\hypersetup{colorlinks=true,%
	citecolor=red,%
	linkcolor=red,%
	urlcolor=blue,%
	pdftex}
	
\newcommand{\opensource}{\textit{open source}}
\newcommand{\software}{\textit{software}}
\newcommand{\calopsita}{Calopsita}

\title{PARTE SUBJETIVA}
\author{Cauê Haucke Porta Guerra}

\begin{document}

\maketitle

\newpage
Entrei no IME meio que por acaso. Fiz cursinho para ingressar na Academia da Força Aérea para me tornar um Oficial Aviador. Depois de um mês de rotina militar acabei desistindo daquela carreira e, por sorte, o IME estava realizando a matrícula de alunos oriundos da 3a. chamada, justo a que eu havia passado. Me matriculei então no BMAC e consegui uma transferência para o BCC na última chamada -- fui o último aluno a entrar no curso. Não me arrependo dessa mudança toda. Hoje, a formação que o IME me proporcionou vai muito além dos 199 créditos necessários para obter o tão sonhado diploma. Cresci como pessoa e como profissional e cada dia dentro desse instituto contribuiu para isso.

O \calopsita{} é justamente a consolidação e a apresentação de todo esse conhecimento adquirido e me sinto realmente orgulhoso em ver o resultado que obtivemos e o aprendizado que foi gerado. Reconheço também que o projeto só teve o resultado obtido com a junção dos conhecimentos adquiridos na faculdade com aqueles adquiridos no estágio na Caelum.

\section{Desafios e frustrações}

Um grande desafio para mim em relação ao BCC foi em conciliar o estudo necessário para aproveitar as disciplinas cursadas com estágios e trabalhos. Para mim, era importante financeiramente poder trabalhar desde o primeiro semestre de minha graduação e a falta de equilíbrio nessas duas atividades acabaram me causando algumas reprovações. Além disso, algumas más escolhas em relação a quais disciplinas cursar em determinados semestres também me fizeram atrasar um pouco a conclusão do curso.

Uma frustração, que serviu de grande aprendizado, foi em ter feito uma má escolha em relação ao meu trabalho de conclusão de curso no ano anterior. Havia escolhido um projeto já em andamento, com um código legado sem testes e com uma arquitetura ruim. Toda e qualquer modificação que eu precisava fazer era um ato de coragem: não conseguia garantir que todo o resto continuava funcionando. Isso naturalmente me desmotivou e acabei por abandonar o projeto e adiar de forma definitiva minha formatura em um ano. Isso acabou gerando um desafio também, o de escolher um tema interessante e um grupo motivado o bastante para fazer tudo dar certo.

Em relação ao \calopsita{}, um desafio para mim foi fazer com que nosso sistema tivesse a usabilidade que nossos clientes esperavam. Isso envolvia bastante código em \textit{javascript}, que eu pouco conhecia, então muito estudo foi necessário. Uma frustração em relação aos códigos \textit{javascript} é que nenhum de nós sabia direito como testá-los de forma unitária e acabamos por demorar muito para começar esses testes e por conta disso, algumas mudanças quebravam todo o resto.

\section{Disciplinas cursadas mais relevantes}

Eu diria que todas as disciplinas cursadas foram fundamentais na realização desse trabalho e na formação do profissional que sou hoje. Mesmo o \calopsita{} sendo um sistema, as matérias mais teóricas e matemáticas tiveram sua importância na formação da base necessária e no formalismo exigidos de um cientista da computação. No entanto, as que influenciaram mais diretamente na realização desse trabalho foram:

\begin{itemize}
	\item{\textbf{MAC0110 - Introdução à Computação} - importante por ser o primeiro contato dos alunos com computação. Serve de base para todo o resto.}
	\item{\textbf{MAC0122 - Princípios de Desenvolvimento de Algoritmos} - junto com MAC0110, forma a base de todo o conhecimento que será adquirido durante o curso.}
	\item{\textbf{MAC0211 - Laboratório de Programação I} - a importância dessa disciplina pra mim foi o aprendizado de \LaTeX{} e sistemas de controles de versão, além de ser a primeira disciplina onde tivemos de trabalhar em um projeto de duração um pouco maior. Foi a primeira disciplina onde fiz um projeto mais próximo a um sistema de verdade. Fizemos um jogo e fomos responsáveis por toda a sua concepção, jogabilidade, inteligência artificial, parte gráfica. Foi quando eu percebi que já era capaz de juntar todo o conhecimento adquirido até então para programar algo completo, do início ao fim.}
	\item{\textbf{MAC0332 - Engenharia de Software} - o projeto que tivemos que desenvolver  foi o primeiro de que participei com uma equipe grande, cerca de 10 pessoas. Foi também um projeto onde experimentei algumas frustrações com pessoas que não participavam em nada, o que me ensinou que mais importante que código são as pessoas envolvidas.}
	\item{\textbf{MAC0342 - Laboratório de Programação Extrema} - para mim essa foi a matéria mais importante de todo o curso. Foi onde eu dei meus primeiros passos em programação orientada a testes, pareamento, métodos ágeis. Foi uma ruptura em relação ao que vinha usando no meu estágio até então (trabalhava em bancos de investimentos) e me fez abrir os olhos para novas formas de pensar. Um ponto ruim, foi que o projeto que eu escolhi para participar tinha uma série de problemas em relação a cobertura de testes e modularização de código e isso tornou o desenvolvimento algo menos prazeiroso e menos produtivo do que poderia ter sido.}
	\item{\textbf{MAC0414 - Linguagens Formais e Autômatos} - essa disciplina não teve muito relacionamento com o \calopsita{}. No entanto, acredito ser de fundamental importância. Hoje tenho trabalhado bastante em sistemas RESTful utilizando maquinas de estado, que nada mais são do que autômatos.}
	\item{\textbf{MAC0441 - Programação Orientada a Objetos} - mostrou os principais padrões de projeto. Buscamos seguir esses padrões e boas práticas em todo o desenvolvimento do \calopsita{}.}
\end{itemize}

Ainda gostaria de citar o apoio do IME a diversas atividades extra-curriculares, em especial ao patrocínio de caravanas ao FISL (Forum Internacional de Software Livre), que propiciaram meu primeiro contato com a comunidade de desenvolvimento de \software{}.

\section{Futuro}

Pretendo continuar trabalhando no \calopsita{} e em todos os demais projetos \opensource{} com os quais me envolvi. Pretendo ainda continuar com o foco em metodologias e métodos ágeis. Um resultado de tanto esforço e dedicação, é que logo após a conclusão desse curso devo me mudar para a Austrália, para trabalhar em uma empresa que é referência na utilização de métodos ágeis em projetos mundo afora.

\section{Agradecimentos}
Não tenho como concluir esse curso sem agradecer algumas pessoas. Meu avô, Marcio José Porta merece grande destaque em toda minha formação. Foi ele o responsável pelos meus primeiros passos em programação e quem sempre me incentivou torcendo a cada passo. Aos meus pais, por toda minha ausência e pelo apoio recebido durante os momentos de estudo e cansaço. Agradeço também a meus amigos-clientes, Mariana Bravo e Hugo Corbucci, por terem encarado esse desafio conosco. Eles certamente também são responsáveis pelo sucesso do projeto. Agradeço ao nosso orientador, Prof. Dr. Alfredo Goldman, pelo apoio, críticas e envolvimento. Agradeço aos irmãos Guilherme e Paulo Silveira, pelo patrocínio do projeto, pelos conselhos, pela amizade, pela oportunidade de trabalhar na Caelum. Por fim, agradeço aos meus dois grandes amigos Cecilia Fernandes e Lucas Cavalcanti, com quem tenho o enorme prazer de poder trabalhar nesse projeto e em tantos outros.

\end{document}
