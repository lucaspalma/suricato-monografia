\section{Introdução}
Times ágeis são aqueles que seguem e otimizam os preceitos descritos no Manifesto Ágil~\cite{manifesto}. Esses times se preocupam em atender às expectativas do usuário, ainda que essas mudem no decorrer do projeto, além de melhorar a qualidade do trabalho executado pela equipe.

Um dos princípios definidos no manifesto discorre sobre a busca pela melhoria contínua das equipes:

\begin{quote}
	\textit{``At regular intervals, the team reflects on how to become more effective, then tunes and adjusts its behavior accordingly.''}~\cite{manifesto}
\end{quote}

 Para promover essa evolução, uma das práticas mais adotadas pelos times ágeis é a retrospectiva. Segundo Derby e Larsen~\cite{retrospectives}, retrospectiva é uma reunião especial em que o time se junta para investigar e melhorar seus métodos - um tempo dedicado à aprendizagem, que funciona como um catalisador de mudanças e ações. Em contraste às tradicionais lições aprendidas, ela foca tanto no processo de desenvolvimento, quanto nas questões humanas do projeto.

Desta forma, a retrospectiva cria um ambiente propício para pessoas ouvirem diferentes percepções e pontos de vista e, assim, ampliarem a visão sobre as diversas situações pelas quais o time passa. Essa reunião é, portanto, uma importante ferramenta, já que dá a oportunidade de entender como melhorar suas habilidades, a produtividade do time e a qualidade do produto para cada membro.

No entanto, por mais que os times ágeis conheçam a reunião e entendam seu funcionamento, a aplicação da teoria nem sempre apresenta bons resultados. Logo, o \suricato{} é um projeto que surgiu da ideia de ajudar times ágeis a realizarem retrospectivas com mais facilidade.

%TODO : revisar os parágrafos com a Ceci
A necessidade da aplicação surgiu na equipe em que ambos os idealizadores do projeto trabalham. Problemas como juntar todos os membros em um único dia, manter o foco das pessoas e organizar a reunião ocorriam de tempos em tempos e mesmo buscando e aplicando possíveis soluções, alguns desses problemas retornavam.

No entanto, a intenção do projeto não é substituir a retrospectiva presencial, mas prover uma ferramenta que  ajude os times ágeis a alcançarem a melhoria contínua. Além disso, o \suricato{} guarda históricos das reuniões para saber quais são as atividades realizadas com maior frequência pelos times e, eventualmente, sugerir para seus usuários novos formatos para a reunião.   

Esse texto está organizado da seguinte forma: na seção 2, explicamos a relação entre melhoria contínua e retrospectivas, assim como sua estrutura sugerida. Em seguida, na seção 3, descrevemos a pesquisa que foi submetida à comunidade ágil brasileira e os resultados obtidos. Na seção 4, explicamos o processo de construção do projeto Suricato, que visa auxiliar times a resolver os problemas levantados na pesquisa e, por fim, concluímos na seção 5 com nossos pareceres e expectativas para o futuro do projeto e da pesquisa.