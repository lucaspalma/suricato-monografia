\section{Respostas da pesquisa}
\label{respostas}

A pesquisa teve 44 respostas, sendo que 24 pessoas responderam todas as perguntas e 20 apenas a primeira página.

\subsubsection*{Pessoas que responderam apenas a página 1}

\newenvironment{respostas1}[4] {
		\begin{tabular}{| c | m{14em} |}
		\hline
		Como a maioria do seu time é estruturado? & #1 \\
		\hline
		Quais problemas seu time enfrenta?        & #2 \\
		\hline
		Seu time possui quantos integrantes?      & #3 \\
		\hline
		Seu time realiza retrospectivas?          & #4 \\	
		\hline
		\end{tabular}
	
}

\begin{enumerate}
	\item
	\begin{respostas1}
		{Local}
		{Comunicação e relacionamento}
		{1 - 10}
		{Não}
	\end{respostas1}

	\item
	\begin{respostas1}
		{Local}
		{Nenhum}
		{1 - 10}
		{Não}
	\end{respostas1}

	\item
	\begin{respostas1}
		{Local}
		{Nenhum}
		{1 - 10}
		{Sim}
	\end{respostas1}

	\item
	\begin{respostas1}
		{Distribuído no mesmo timezone}
		{Produtividade}
		{1 - 10}
		{Não}
	\end{respostas1}

	\item
	\begin{respostas1}
		{Distribuído no mesmo timezone}
		{Produtividade}
		{21 - 30}
		{Não}
	\end{respostas1}

	\item
	\begin{respostas1}
		{Local}
		{Comunicação}
		{1 - 10}
		{Não}
	\end{respostas1}

	\item
	\begin{respostas1}
		{Local}
		{Comunicação e produtividade}
		{mais de 30}
		{Sim}
	\end{respostas1}

	\item
	\begin{respostas1}
		{Local}
		{Comunicação, relacionamento e produtividade}
		{1 - 10}
		{Não}
	\end{respostas1}

	\item
	\begin{respostas1}
		{Local}
		{Produtividade}
		{1 - 10}
		{Sim}
	\end{respostas1}

	\item
	\begin{respostas1}
		{Local}
		{Comunicação e produtividade}
		{11 -20}
		{Não}
	\end{respostas1}

	\item
	\begin{respostas1}
		{Distribuído no mesmo timezone}
		{Comunicação}
		{1 - 10}
		{Sim}
	\end{respostas1}

	\item
	\begin{respostas1}
		{Distribuído no mesmo timezone}
		{Comunicação}
		{11 - 20}
		{Sim}
	\end{respostas1}

	\item
	\begin{respostas1}
		{Local}
		{Comunicação e produtividade}
		{11 - 20}
		{Não}
	\end{respostas1}

	\item
	\begin{respostas1}
		{Local}
		{Comunicação e produtividade}
		{1 - 10}
		{Sim}
	\end{respostas1}

	\item
	\begin{respostas1}
		{Local}
		{Comunicação}
		{11 - 20}
		{Sim}
	\end{respostas1}

	\item
	\begin{respostas1}
		{Local}
		{Produtividade}
		{1 - 10}
		{Sim}
	\end{respostas1}

	\item
	\begin{respostas1}
		{Local}
		{Comunicação e produtividade}
		{11 - 20}
		{Sim}
	\end{respostas1}

	\item
	\begin{respostas1}
		{Distribuído no mesmo timezone}
		{Produtividade}
		{1 - 10}
		{Sim}
	\end{respostas1}

	\item
	\begin{respostas1}
		{Local}
		{Nenhum}
		{1 - 10}
		{Sim}
	\end{respostas1}

	\item
	\begin{respostas1}
		{Local}
		{Nenhum}
		{1 - 10}
		{Sim}
	\end{respostas1}
\end{enumerate}


\subsubsection*{Respostas completas - times que fazem retrospectiva}

\newenvironment{respostas2}[9] {
		\begin{tabular}{|m{15em}|m{25em}|}
		\hline
		Como a maioria do seu time é estruturado?  & #1 \\
		\hline
		Quais problemas seu time enfrenta?         & #2 \\
		\hline
		Seu time possui quantos integrantes?       & #3 \\
		\hline
		Com que frequência realiza retrospectivas? & #4 \\	
		\hline
		Vocês variam as atividades e/ou formato das suas retrospectivas? Se sim, cite alguns.											& #5 \\
        \hline
		Quem participa das retrospectivas?         & #6 \\
         \hline
		Quais desafios seu time enfrenta ao realizar uma retrospectiva?
												   & #7 \\
        \hline
		Vocês utilizam algum site, programa ou ferramenta para ajudar a realizar a retrospectiva? Se sim, cite alguns.        & #8 \\
        \hline
		Alguém fica responsável por ser o facilitador da retrospectiva?
											       & #9 \\
		\hline
		\end{tabular}
}

\begin{enumerate}
	\item
	\begin{respostas2}
		{Local}
		{Produtividade}
		{1 - 10}
		{15 dias}
		{Não}
		{Time de desenvolvimento}
		{Agendar dia, horário e lugar \newline Engajamento das pessoas}
		{Não}
		{Líder técnico}
	\end{respostas2}

	\item
	\begin{respostas2}
		{Local}
		{Comunicação e produtividade}
		{11 - 20}
		{1x a cada 15 dias}
		{Sim. Papel pontos positivos e negativos, reuniões para coversar sem post its,}
		{Todo o time}
		{Discussões de pouco valor \newline Engajamento das pessoas}
		{Não}
		{O facilitador varia, cada retro é uma pessoa}
	\end{respostas2}

	\item
	\begin{respostas2}
		{Local}
		{Nenhum}
		{1 - 10}
		{A cada 2 semanas}
		{Não}
		{Todo o time: scrum master, Dev team e PO}
		{Ultrapassar a duração}
		{Atualmente PO organiza tudo em Excel}
		{Não}
	\end{respostas2}

	\item
	\begin{respostas2}
		{Local}
		{Comunicação, relacionamento e produtividade}
		{1 - 10}
		{A cada 2 semanas}
		{Sim; learning matrix, happiness radar, startfish, etc}
		{Scrum master e time}
		{Discussões de pouco valor \newline Engajamento das pessoas}
		{Não}
		{Sim, mas não é do time}
	\end{respostas2}

	\item
	\begin{respostas2}
		{Local}
		{Outro: Um projeto grande com atividades menores de manutenção de outros projetos. Alternância de foco}
		{1 - 10}
		{A cada 3 semanas}
		{Não}
		{Todo o time de desenvolvimento. Quando é possível tentamos envolver o cliente.}
		{Ultrapassar a duração \newline Engajaento das pessoas}
		{Não}
		{Líder}
	\end{respostas2}

	\item
	\begin{respostas2}
		{Distribuído no mesmo timezone}
		{Comunicação e produtividade \newline Outros: Dependências Não-Gerenciadas}
		{11 - 20}
		{A cada 2 semanas}
		{Sim, variamos! Basicamente elas seguem o formato sugerido pela Esther Derby e Diana Larsen (Agile Retrospectives), mas não linearmente. Utilizo técnicas para: - Criar ambiente seguro para os participantes tocarem em assuntos delicados (working agreements) - Setar o contexto, pedindo que falem um sentimento para a Sprint atual, ou que desenhem como foi a Sprint atual, ou que escrevam em post-its (dependendo do nível de introspectividade do grupo) - Pergunto, em casos de falha, por que a Sprint falhou - Coloco, em casos de sucesso, quais fatores nos fizeram chegar lá, e o que poderí­amos fazer pra chegarmos lá ainda melhores ou mais cedo - Utilizo um espaço de trabalho informativo, deixando visíveis, board, burndown, meta da Sprint, Definição de Pronto e outros artefatos para iluminar as ideias e trazer possiveis pontos de melhoria - Projeto o burndown maior para fazermos uma timeline em cima dele e dos acontecimentos que ele projeta - Traçamos planos de ação em forma de brainstorming falado ou escrito - Utilizo tecnicas de priorização quando há muitos pontos a serem tratados pelos participantes - Faço o time se apreciar pelo apoio mútuo em atingir a meta - E muitas outras :P}
		{Time, Product Owner, Scrum Master, Agile Coach.Convidados em caso especial, com consentimento dos três acima.}
		{Falta de intimidade entre integrantes \newline Engajamento das pessoas \newline Falta de anonimato}
		{Não}
		{Agile coach ou Scrum master}
	\end{respostas2}

	\item
	\begin{respostas2}
		{Local}
		{Comunicação}
		{1 - 10}
		{A cada fim de sprint (15 dias de interação)}
		{Variamos de acordo com a necessidade. O padrão é fazer um formato de retrospectiva. Mas as vezes temos pouco tempo, ou o sprint teve poucos problemas ai fazemos outro formato. O padrão de fazer a matriz (começar, continuar, melhorar, parar). O formato simples e rápido é colocarmos apenas os assuntos que aconteceram nos post its e ações para melhoras no próximo sprint. Quando o clima esta legal e temos tempo, fazemos outros formatos mais lúdicos, como timeline.}
		{Todos! Po, Time e SM. Ninguém fica de fora para não ter ruido na comunicação e fofoca.}
		{Ultrapassar a duração}
		{Acima já coloquei alguns nos links. Mas como SM sempre tenho o.modelo de PDCA na cabeça. Ou seja, sempre tem que ter melhorias.}
		{Sim, organiza reuniões, facilita conversas, cuida do quadro, auxilia o PO com o Backlog, alinha expectativas com stakeholders, remove impedimentos e acompanha o desenvolvimento individual de cada integrante da equipe!}
	\end{respostas2}

	\item
	\begin{respostas2}
		{Local}
		{Comunicação e produtividade}
		{1 - 10}
		{A cada fim de sprint (2 semanas)}
		{Sim, dependendo do que queremos mostrar.; Team building com this guy/that guy, timelines e do's and don'ts, starfish, (stop/start/continue), dinamicas de apreciação, etc. sempre atendo-se ao formato de "warm-up, coleta de dados, discussão e ações". para cada ação que levantamos, colocamos um dono dela, que será responsável por fazê-la acontecer}
		{Apenas os membros do time}
		{Discussão de pouco valor \newline Engajamento das pessoas}
		{Não}
		{Sim}
	\end{respostas2}

	\item
	\begin{respostas2}
		{Local}
		{Comunicação e produtividade \newline Outros: capacitação}
		{1 - 10}
		{A cada sprint}
		{Não}
		{Todo o time, mais um integrante de uma equipe separada responsável por Qualidade de Processos (uma espécie de "escritório" de SMs).}
		{Ultrapassar a duração \newline Discussão de pouco valor \newline Falta de intimidade integrantes \newline Engajamento das pessoas}
		{Um quadro. Pra organizar os itens que foram bons, os que precisam melhorar e as ações de melhoria.}
		{Scrum master}
	\end{respostas2}

	\item
	\begin{respostas2}
		{Local}
		{Produtividade}
		{1 - 10}
		{Ao final de cada sprint}
		{Não}
		{Scrum Team}
		{Engajamento das pessoas \newline Outro: Olham como obrigação e não como um momento para a melhoria do processo.}
		{Não}
		{Scrum master}
	\end{respostas2}

	\item
	\begin{respostas2}
		{Local}
		{Comunicação, relacionamento e produtividade}
		{1 - 10}
		{Fim de projeto}
		{Não}
		{Todo time}
		{Ultrapassar a duração}
		{Não}
		{Sim, mas não é do time}
	\end{respostas2}

	\item
	\begin{respostas2}
		{Distribuído em diversos timezones}
		{Comunicação}
		{21 - 30}
		{Final de cada sprint}
		{Não}
		{O time, PO e Gps}
		{Discussão de pouco valor}
		{Não}
		{Sim, o gp}
	\end{respostas2}

	\item
	\begin{respostas2}
		{Distribuído no mesmo timezone}
		{Nenhum}
		{11 - 20}
		{Mensal}
		{Não}
		{Todo o time}
		{Ultrapassar a duração \newline Discussões de pouco valor \newline Falta de intimidade entre integrantes}
		{Não}
		{PO}
	\end{respostas2}

	\item
	\begin{respostas2}
		{Distribuído no mesmo timezone}
		{Comunicação e relacionamento}
		{1 - 10}
		{Postmortem}
		{Não}
		{Todos os integrantes do time}
		{Discussões de pouco valor \newline Falta de intimidade entre entegrantes}
		{Não}
		{Líder ou gerente}
	\end{respostas2}

	\item
	\begin{respostas2}
		{Local}
		{Produtividade}
		{1 - 10}
		{Quinzenal}
		{Sim. Usamos algumas técnicas: fishbowl, morte do produto, brainstorm, brainstorm reverso, hot-air ballon, 6 chapéus e outros.}
		{Time, Scrum Master e PO}
		{Ultrapassar a duração}
		{Fun retrospectives}
		{Scrum master ou Dev}
	\end{respostas2}

	\item
	\begin{respostas2}
		{Distribuído no mesmo timezone}
		{Nenhum}
		{1 - 10}
		{Quinzenal}
		{Sim}
		{Todos do time, PO, Devs, tester, suporte.}
		{Ultrapassar a duração \newline Outro: perder o foco}
		{Não}
		{Revesado, todos um dia é facilitador.}
	\end{respostas2}

	\item
	\begin{respostas2}
		{Distribuído no mesmo timezone}
		{Comunicação}
		{11 - 20}
		{Semanalmente}
		{Discutimos o que está ruim e precisamos mudar, o que foi bom e precisamos continuar. Além disso sempre discutimos alguns assuntos que necessitam da presença de todos.; Vez por outra fazemos algumas práticas como: o Fishbowl.}
		{Ao total dura entre 1h e 1:30h. Temos 2 times: operações e desenvolvimento. nos primeiros 30min cada time faz sua própria retro para alinhamento de quesitos que compete somente a cada time. Depois os times se reúnem e abordam questões que envolvem os dois times.}
		{Ultrapassar a duração \newline Outro: Por vezes é necessário alguém da diretoria na reunião. O que acontece de vez enquando.}
		{Utilizamos apenas o quadro branco. Quando é necessário usamos folhas e canetas para anotações ....}
		{É aleatório ... lidera quem quer sempre respeitando para não ser sempre o mesmo}
	\end{respostas2}

	\item
	\begin{respostas2}
		{Local}
		{Produtividade}
		{1 - 10}
		{Sempre}
		{Não}
		{Dev, PO e SM}
		{Ultrapassar a duração}
		{Não}
		{Não}
	\end{respostas2}

	\item
	\begin{respostas2}
		{Local}
		{Produtividade}
		{1 - 10}
		{Todo sprint, que tem duração de 4 semanas}
		{Não}
		{somente time de desenvolvimento}
		{Discussão de pouco valor \newline Falta de engajamento}
		{Não}
		{Scrum master}
	\end{respostas2}
\end{enumerate}

\subsubsection*{Respostas completas - times que não fazem retrospectiva}

\newenvironment{respostas3}[8] {
		\begin{tabular}{|m{15em}|m{25em}|}
		\hline
		Como a maioria do seu time é estruturado?  & #1 \\
		\hline
		Quais problemas seu time enfrenta?         & #2 \\
		\hline
		Seu time possui quantos integrantes?       & #3 \\
		\hline
		De que forma os problemas que seu time já tem são discutidos e resolvidos? 
												   & #4 \\	
		\hline
		De que forma seu time identifica problemas futuros e trabalha para evitá-los? 												   & #5 \\
        \hline
		Quem participa das discussões sobre os problemas do time? E quem resolve os problemas? 							       & #6 \\
         \hline
		Quais os motivos para seu time não fazer retrospectivas? 
												   & #7 \\
        \hline
		Com que frequência o time inteiro se reúne, seja local ou virtualmente? 
											       & #8 \\
		\hline
		\end{tabular}
}

\begin{enumerate}
	\item
	\begin{respostas3}
		{Distribuído em timezones diversos}
		{Comunicação e produtividade}
		{11 - 20}
		{Chat direto entre a pessoa que está problema e a pessoa que pode resolve-lo.}
		{Não há um meio formal para tal. Temos chats no Skype que usamos pra discutir dúvidas, soluções, sugestões, etc. Cada sub-time (front, back, db) possui um chat, e um chat entre os "cabeças" de cada sub-time, e um chat com o time todo. São criados chats sob demanda também.}
		{Não há alguém encarregado, normalmente todas as opiniões são escutadas e discutidas, e tentamos seguir a melhor solução.}
		{Má gerência. Nosso gerente é muito inexperiente com essas metodologias. De pouco em pouco estamos corrigindo isso.}
		{Todos os dias, via chat.}
	\end{respostas3}

	\item
	\begin{respostas3}
		{Distribuído em timezones diversos}
		{Comunicação \newline Outros: sincronização}
		{11 - 20}
		{Conference as por telefone ou online}
		{Falhadas Rapido Melhor do que falhar tarde, assim mais facil identificar o problema em algo}
		{Todos is envolvidos. Depende da complex idled e importancia fermented Sao envolvidos}
		{Sem motivos aparentes, Altas quabtidades de demandas para time restrito em quantidade de recursos}
		{cada celula tem reunions diarias de sync e o time inteiro se reune a Casa Laguna meses}
	\end{respostas3}

	\item
	\begin{respostas3}
		{Distribuído em timezones diversos}
		{Relacionamento \newline Outros: Equipes descentralizadas e pouco integradas}
		{Mais de 30}
		{Reuniões informações e sem hora prevista}
		{Por demanada.}
		{Líderes de projetos junto com a equipe.}
		{Atualmente os prazos estão comprometidos junto aos cliente (entregas). Por isso temos tido pouco tempo para executar reunioes de verificação.}
		{Quase nunca. Muita gente}
	\end{respostas3}

	\item
	\begin{respostas3}
		{Distribuído em timezones diversos}
		{Relacionamento}
		{Mais de 30}
		{Em reuniões com os integrantes envolvidos no problema, caso o problema seja de utilidade para todo o time, a solução é repassada posteriormente ao time todo.}
		{De uma forma geral, não há esse tipo de preocupação.}
		{Quem está envolvido no problema e os responsáveis pelo produto, como product owner, analistas, diretores, etc.}
		{Não sei exatamente, decisão dos gestores do projeto.}
		{Diariamente.}
	\end{respostas3}

	\item
	\begin{respostas3}
		{Distribuído em timezones diversos}
		{Comnicação \newline Outros: Agilidade para corrigir bug ou entrar novas funcionalidades quando para concluir a tarefa precisa do envolvimento de outros membros da equipe}
		{1 - 10}
		{Muitas das vezes tentamos realizar um call com todos os membros de forma que o timezone fica fácil para todos. Quando identificamos os problemas cada um da sua opinião de como podemos melhorar a situação e o Gerente de Projetos se encarrega de seguir a opinião que decidimos.}
		{O time é muito focado na parte do código, então quando achamos que pode acontecer algum problema em certo ponto da aplicação, colocamos um alerta nesse ponto para que o time olhe e decida qual o melhor caminho para realizar.}
		{Todos os membros da equipe, normalmente fica a cargo do Gerente de projetos e lí­der de equipe.}
		{não é uma cultura da empresa, nem da equipe. Acaba não dando atenção neste aspecto.}
		{3-4 vezes por mês}
	\end{respostas3}

\end{enumerate}